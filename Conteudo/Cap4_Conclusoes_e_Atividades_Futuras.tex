\chapter{Conclusões e Trabalhos futuros} \label{sec:conclusao}
Neste relatório, foram detalhados todos os métodos, pesquisas e atividades conduzidas no intervalo de tempo de novembro/2019 a outubro/2020 referentes ao desenvolvimento da aplicação \crossword.

Em síntese, neste período foi construído o aplicativo \crossword, que tem o objetivo de estimular o ensino e aprendizagem em idosos por meio de palavras cruzadas em conhecimentos gerais. Ainda, foi disponibilizado o conteúdo em formato de texto, áudio e vídeo a fim de complementar a experiência do estudo.

Ademais, foram estudados conceitos de aprendizagem móvel, aprendizagem voltada para idosos e, com maior aprofundamento, artefatos e diretrizes de acessibilidade. Também, foram utilizados no desenvolvimento da aplicação \crossword a metodologia Scrum; realizou-se a análise de aspectos técnicos do \textit{React Native}, \textit{NodeJS}, \textit{MongoDB} e \textit{Docker}. A fim de complementar o conhecimento base para a construção do aplicativo, foi feito um minicurso de interface e experiência do usuário (UI/UX).

Três avaliações foram feitas em etapas diferentes do projeto. A primeira foi voltada à tela do jogo de palavras cruzadas e seu aperfeiçoamento. A segunda, por especialistas visando melhoras e utilizando as heurísticas de Nielsen. A terceira e última, por uma especialista no ensino de idosos com o objetivo de reparar os últimos detalhes para a publicação do aplicativo.

Vale ressaltar que durante o desenvolvimento, o projeto foi submetido no 28º Siicusp (Simpósio Internacional de Iniciação Científica e Tecnológica da USP) no ano de 2020.

Como trabalhos futuros, ressalta-se a necessidade de  disponibilizar a ferramenta nas lojas de aplicativos Play Store e App Store, pois aumentaria o potencial de disseminação e adesão. É importante destacar que o setor de educação tem altas taxas de \textit{download} em ambas as lojas\footnote{\href{https://www.statista.com/statistics/270291/popular-categories-in-the-app-store/}{App Store};  \href{https://www.statista.com/statistics/279286/google-play-android-app-categories/}{Play Store}}. Dessa maneira, seria possível criar uma base de usuários sólida e recorrente ao gerar valor para o público alvo. Além disso, eles receberiam atualizações de futuras mudanças e correções de problemas.

Pretende-se ainda, construir uma plataforma de gestão para os professores. Nela, eles poderiam criar, editar e apagar disciplinas e aulas com conteúdos em texto, áudio e vídeo. Além disso, um dos objetivos é integrar o algoritmo de criação de palavras-cruzadas à plataforma, a fim de facilitar sua construção e inserção no banco de dados para acesso pelos usuários por meio do aplicativo.

Finalmente, é possível concluir que o projeto atendeu às expectativas iniciais e o produto final é válido para ser utilizado por professores e alunos em conjunto com a sala de aula. 
Ainda, espera-se que cause impacto na comunidade de aprendizes idosos de todas as regiões, ajudando na difusão da educação no país.
