\chapter{Trabalhos futuros} \label{sec:atividades_futuras}
Como trabalhos futuros, ressalta-se a necessidade de  disponibilizar a ferramenta nas lojas de aplicativos Play Store e App Store, pois aumentaria o potencial de disseminação e adesão. É importante destacar que o setor de educação tem altas taxas de \textit{download} em ambas as lojas\footnote{\href{https://www.statista.com/statistics/270291/popular-categories-in-the-app-store/}{App Store};  \href{https://www.statista.com/statistics/279286/google-play-android-app-categories/}{Play Store}}. Dessa maneira, seria possível criar uma base de usuários sólida e recorrente ao gerar valor para o público alvo. Além disso, eles receberiam atualizações de futuras mudanças e correções de problemas.

Pretende-se ainda, construir uma plataforma de gestão para os professores. Nela, eles poderiam criar, editar e apagar disciplinas e aulas com conteúdos em texto, áudio e vídeo. Além disso, um dos objetivos é integrar o algoritmo de criação de palavras-cruzadas à plataforma, a fim de facilitar sua construção e inserção no banco de dados para acesso pelos usuários por meio do aplicativo.

Falar aqui sobre a colaboração e inserção na tese de doutorado da Camila.