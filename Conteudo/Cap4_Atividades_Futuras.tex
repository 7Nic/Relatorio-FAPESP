\chapter{Atividades Previstas e Cronograma} \label{sec:atividades_futuras}
Como atividades futuras, pode-se destacar:
\begin{itemize}
    % \item Estudar a \reqmlcatalog e elicitar requisitos para desenvolvimento
    % \item Estudar padrões, linguagens de padrões e \mlearningpl
    % \item Elicitação de requisitos/criação de histórias do usuário com base na \mlearningpl
    % \item Propor interfaces baseadas nas diretrizes estudadas e desenvolver o aplicativo
    % \item Realizar testes com usuários idosos a fim de propor melhorias de uso
    \item Estudo dos conceitos de acessibilidade: esta atividade consistirá no estudo e entendimento dos principais conceitos referentes à acessibilidade.
    \item Estudo dos conceitos de ensino e aprendizagem para usuários idosos: esta atividade consistirá no estudo e entendimento dos principais conceitos referentes ao processo de ensino e aprendizagem considerando usuários idosos.
    \item Estudo de artefatos: esta atividade consistirá no estudo de artefatos que apoiem o processo de desenvolvimento de aplicações educacionais móveis, tais como catálogo de requisitos, linguagem de padrões, modelo de qualidade, entre outros.
    \item Definição de requisitos da \crossword: esta atividade consistirá na elicitação dos requisitos da aplicação educacional móvel a ser desenvolvida.
    \item Projeto da \crossword: esta atividade consistirá na modelagem da aplicação educacional móvel.
    \item Desenvolvimento da \crossword: esta atividade consistirá na implementação da aplicação móvel.
    \item Avaliação da \crossword: esta atividade consistirá na condução de estudos empíricos para avaliar a aplicação móvel desenvolvida. As avaliações considerarão os aspectos pedagógicos e de usabilidade e acessibilidade da aplicação.
    \item Elaboração de artigos e relatórios: esta atividade registrará as etapas conduzidas durante o desenvolvimento do trabalho e os resultados obtidos a partir dessa experiência. Os artigos desenvolvidos serão submetidos a congressos de iniciação científica e outros congressos nas áreas de interesse.
\end{itemize}

\begin{table}[!ht]
\centering
\caption{\textit{Cronograma de atividades}}
\label{tab:cronograma}
\begin{tabular}{|c|c|c|c|c|c|c|c|c|c|c|c|c|}
\hline
\multirow{2}{*}{\textbf{Atividades}} & \multicolumn{12}{c|}{\textbf{Meses}} \\ \cline{2-13} 
 & 1 & 2 & 3 & 4 & 5 & 6 & 7 & 8 & 9 & 10 & 11 & 12 \\ \hline
1) Conceitos - usabilidade e acessibilidade & $\checkmark$ & $\checkmark$ &  &  &  &  &  &  &  &  &  &  \\ \hline
2) Conceitos - aprendizagem móvel & $\checkmark$ & $\checkmark$ &  &  &  &  &  &  &  &  &  &  \\ \hline
3) Conceitos - ensino idosos & $\checkmark$ & $\checkmark$ &  &  &  &  &  &  &  &  &  &  \\ \hline
4) Artefatos &  & $\checkmark$ & $\checkmark$ & $\checkmark$ &  &  &  &  &  &  &  &  \\ \hline
5) Requisitos &  &  & $\bullet$ & $\bullet$ & $\bullet$ &  &  &  &  &  &  &  \\ \hline
6) Projeto &  &  &  & $\checkmark$ & $\checkmark$ & $\checkmark$ &  &  &  &  &  &  \\ \hline
7) Levantamento e estudo de tecnologias &  &  &  &  & $\checkmark$ & $\checkmark$ &  &  &  &  &  &  \\ \hline
8) Desenvolvimento &  &  &  &  & $\checkmark$ & $\checkmark$ & $\bullet$ & $\bullet$ & $\bullet$ & $\bullet$ &  &  \\ \hline
9) Avaliação &  &  &  &  &  &  &  &  &  & $\bullet$ & $\bullet$ & $\bullet$ \\ \hline
10) Artigos e relatórios &  &  &  &  &  & $\checkmark$ &  &  & $\bullet$ &  &  & $\bullet$ \\ \hline
\end{tabular}
\end{table}