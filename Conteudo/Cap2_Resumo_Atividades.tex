\chapter{Resumo das Atividades Desenvolvidas} \label{sec:resumo_ativ}
Nesta seção, são descritas sucintamente as principais atividades que foram conduzidas pelo aluno no período de novembro/2019 a abril/2020 para alcançar os objetivos deste trabalho:

\begin{description}
\item[1. Estudos sobre usabilidade e acessibilidade:]
A atividade consistiu no estudo teórico dos conceitos de acessibilidade e usabilidade em contextos gerais da sociedade.

\item[2. Estudo sobre Aprendizagem móvel:] Esta atividade consistiu no aprofundamento e estudo dos conceitos de aprendizagem móvel. Foram analisados os principais conceitos relacionados à área, as vantagens e limitações no que tange a aprendizagem móvel. 

\begin{description}
    \item[2.1. Panorama:]
    A atividade foi analisar o panorama da aprendizagem móvel. Do que se trata, vantagens e desvantagens.
    
    \item[2.2. Pesquisa sobre as principais funcionalidade de aplicativos de aprendizagem móvel:]
    Esta atividade consistiu no estudo de funcionalidades presentes em aplicativos de aprendizagem. Foram analisadas diversas plataformas de ensino móvel e decidiu-se explorar as funcionalidades dos principais.
\end{description}

\item[3. Estudos sobre Ensino voltado para idosos:]
A atividade foi composta explicação da necessidade de mudança da abordagem na educação com o uso da educação continuada. Além disso foram abordadas dificuldades do público idoso no aprendizagem bem como possíveis ações para lidar com elas.

\item[4. Estudo sobre artefatos:]
\hfill

\begin{description}
    \item[4.1. Pesquisa sobre recomendações da WCAG 2.1:]
    Esta atividade consistiu no estudo das recomendações da WCAG (\textit{Web Content Accessibility Guidelines} - Orientações para Acessibilidade em Conteúdos Web) para direcionar a construção do aplicativo de acordo com critérios de acessibilidade.
    
    \item[4.2. Req ML Catalog - Catálogo de Requisitos \citep{soad2017reqml}:]
    A atividade consistiu na análise do referido catálogo de requisitos a fim de basear a elaboração e desenvolvimento do aplicativo.
\end{description}


\item[5. Levantamento e estudo dos conceitos e tecnologias para desenvolvimento de aplicações móveis:]
\hfill

\begin{description}
\item[5.1. Estudos sobre Scrum e Requisitos:] Esta atividade consistiu no aprendizado do Scrum, metodologia ágil para gestão e planejamento de projetos de software, bem como no estudo sobre conceitos de requisitos funcionais, não funcionais e de domínio.

\item[5.2. Estudos sobre React Native:] Esta atividade consistiu na preparação técnica relacionada ao \textit{framework} em javascript React Native, utilizado para desenvolver aplicativos móveis compatíveis com as plataformas Android e iOS.

\item[5.3. Estudos sobre UI/UX:] Embora o foco da aplicação seja a aprendizagem móvel, é indispensável conhecer os conceitos de Experiência de Usuários (UX) e Interface de Usuários (UI). Em vista disso, estudos direcionados ao assunto foram realizados a fim de possibilitar um melhor desenvolvimento da aplicação.

\end{description}

\item[6. Desenvolvimento:]
\hfill

\begin{description}
    \item[6.1. Protótipo inicial:]
    A atividade foi composta da análise do protótipo anteriormente construído com o objetivo de utilizar a pesquisa de validação efetuada.
    
    \item[6.2. Estudo do algoritmo de geração das palavras cruzadas:]
    Esta atividade consistiu no estudo de um algoritmo que pudesse criar um tabuleiro de palavras cruzadas com eficiência utilizando uma pequena lista de palavras fornecidas pelo usuário.
    
    \item[6.3. Desenvolvimento do \textit{MVP}:] 
    A atividade se fundamentou na construção de um produto mínimo viável a fim de realizar outra validação de acessibilidade e usabilidade.
\end{description}

\end{description}