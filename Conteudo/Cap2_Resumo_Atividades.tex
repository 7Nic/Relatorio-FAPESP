\chapter{Resumo das Atividades Desenvolvidas} \label{sec:resumo_ativ}
Nesta seção, são descritas sucintamente as principais atividades conduzidas pelo aluno no período de novembro/2019 a abril/2020 para alcançar os objetivos deste trabalho:

\begin{description}
\item[1. Estudo sobre usabilidade e acessibilidade:]
Esta atividade consistiu no estudo teórico dos conceitos de acessibilidade e usabilidade em contextos gerais da sociedade.

\item[2. Estudo sobre aprendizagem móvel:] Esta atividade consistiu no aprofundamento e estudo dos conceitos sobre aprendizagem móvel. Foram analisados os principais conceitos e características, bem como vantagens e limitações relacionadas à área. 

\begin{description}
    \item[2.1. Visão geral:]
    Esta atividade consistiu em adquirir uma visão geral a respeito da aprendizagem móvel, incluindo conceitos básicos, vantagens e desvantagens associadas.
    
    \item[2.2. Estudo sobre as principais funcionalidades de aplicativos de aprendizagem móvel:]
    Esta atividade consistiu no estudo de funcionalidades presentes em aplicativos de aprendizagem móvel. Foram analisadas diversas plataformas de apoio, sendo investigadas as funcionalidades dos principais delas.
\end{description}

\item[3. Estudo sobre ensino voltado para idosos:] Nesta atividade foram estudados aspectos da educação continuada no contexto de idosos. Também foram abordadas dificuldades do público idoso na aprendizagem bem como possíveis ações para lidar com elas.

\item[4. Estudo de artefatos e requisitos:]
Esta atividade foi composta pelo estudo das diretrizes de acessibilidade denominadas WCAG 2.1. Além disso, para a elicitação de requisitos do aplicativo foram estudados o catálogo ReqML-Catalog \citep{soad2017reqml}, a linguagem de padrões MLearning-PL \citep{Fioravanti2017_plop} e as diretrizes pedagógicas e de acessibilidade para idosos MLGE (\textit{Mobile Learning Guidelines for the Elderly}).

\begin{description}
    \item[4.1. Recomendações da WCAG 2.1:]
    Esta atividade consistiu no estudo das recomendações da WCAG (\textit{Web Content Accessibility Guidelines} -- Orientações para Acessibilidade em Conteúdos Web) para direcionar a construção do aplicativo de acordo com critérios de acessibilidade.
    
    \item[4.2. Requisitos:]
    Esta atividade consistiu no estudo dos requisitos do aplicativo proposto, os quais foram elicitados tendo como base o catálogo ReqML-Catalog (Catálogo de Requisitos para Aplicações Educacionais Móveis), a Linguagem de Padrões MLearning-PL e as diretrizes MLGE.
\end{description}


\item[5. Levantamento e estudo de métodos e tecnologias para desenvolvimento de aplicações móveis:]
Esta atividade consistiu no levantamento e estudo de métodos para o desenvolvimento do aplicativo móvel. Iniciou-se com o estudo da metodologia do Scrum; a seguir foi descrito o \textit{framework} React Native, utilizado no código da aplicação e conceitos importantes de UI/UX para a usabilidade de aplicativo. 

\begin{description}
\item[5.1. Scrum:] Esta atividade consistiu no estudo e entendimento do Scrum \citep{scrumSite}, metodologia ágil para gestão e planejamento de projetos. %, bem como no estudo sobre conceitos de requisitos funcionais, não funcionais e de domínio.

\item[5.2. React Native:] Esta atividade consistiu no estudo e entendimento do \textit{framework} em javascript, React Native \citep{RN}, utilizado para desenvolver aplicativos móveis compatíveis com as plataformas Android e iOS.

\item[5.3. UI/UX:] Para o desenvolvimento de aplicações para aprendizagem móvel, é fundamental que sejam considerados os conceitos de Experiência de Usuários (UX) e Interface de Usuários (UI). Em vista disso, estudos direcionados ao assunto foram realizados a fim de possibilitar um melhor desenvolvimento da aplicação.

\item[5.4 Docker:] Esta atividade consistiu no estudo do programa Docker \citep{docker}, utilizado para criação dos contêineres para a API em NodeJS e para o banco de dados em MongoDB.

\item[5.5 NodeJS:] Esta atividade consistiu no estudo do framework em javascript, NodeJS \citep{nodejs}, o qual foi usado para a infraestrutura do backend.

\item[5.6 MongoDB:] Esta atividade consistiu no estudo do programa de banco de dados não relacional MongoDB \citep{mongodb}.

\end{description}

\item[6. Projeto e Desenvolvimento da \textit{Crossword Learning}:]
Esta atividade consistiu na prototipação e desenvolvimento do aplicativo. Primeiramente foi analisado o protótipo inicial, um \textit{mockup} composto pela interface do aplicativo. Além disso foi estudado um algoritmo que permitisse a construção de um tabuleiro com palavras cruzada. Por fim, foi desenvolvido um MVP (Produto Mínimo Viável -- \textit{Minimum Viable Product})  para a primeira validação.

\begin{description}
    \item[6.1. Protótipo inicial:]
    A atividade consistiu na análise do protótipo construído anteriormente.
    
    \item[6.2. Estudo do algoritmo de geração das palavras cruzadas:]
    Esta atividade consistiu no estudo de um algoritmo que pudesse criar um tabuleiro de palavras cruzadas com eficiência utilizando uma pequena lista de palavras fornecidas pelo usuário.
    
    \item[6.3. Desenvolvimento do \textit{MVP}:] 
    Esta atividade consistiu na construção de um MVP a fim de realizar a validação preliminar de acessibilidade e usabilidade do aplicativo desenvolvido na tela de palavra-cruzada.
    
    \item[6.4. Teste piloto:] 
    Esta atividade consistiu na avaliação preliminar, focando na acessibilidade e usabilidade.
    
    \item[6.5. Desenvolvimento da infraestrutura do aplicativo:] 
    Esta atividade consistiu no desenvolvimento das telas não concernentes à palavra-cruzada, na modelagem e construção do banco de dados e na estruturação da API que funciona como backend da aplicação.
    
    \item[6.6. Avaliação heurística:] 
    Esta atividade consistiu na avaliação do aplicativo por especialistas seguindo as heurísticas de Nielsen.
    
    \item[6.7. Evolução da Crossword Learning:] 
    Esta atividade consistiu na evolução do aplicativo baseada nos comentários obtidos dos especialistas na avaliação heurística.
\end{description}

\end{description}