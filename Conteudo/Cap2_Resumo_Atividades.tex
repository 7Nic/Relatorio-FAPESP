\chapter{Resumo das atividades realizadas} \label{sec:resumo_ativ}
Nesta seção, são descritas sucintamente as principais atividades que foram conduzidas pelo aluno no período de agosto/2019 a outubro/2019 para alcançar os objetivos deste trabalho:

\begin{description}
\item[1. Estudo sobre Aprendizagem móvel:] Esta atividade consistiu no aprofundamento e estudo dos conceitos de aprendizagem móvel. Foram analisados os principais conceitos relacionados à área, as vantagens e limitações no que tange a aprendizagem móvel. 

\item \textbf{2. Levantamento e estudo dos conceitos e tecnologias para desenvolvimento de aplicações móveis:}
 

\begin{description}
\item[2.1. Estudos sobre Scrum e Requisitos:] Esta atividade consistiu no aprendizado do Scrum, metodologia ágil para gestão e planejamento de projetos de software, bem como no estudo sobre conceitos de requisitos funcionais, não funcionais e de domínio.

\item[2.2. Estudos sobre React Native:] Esta atividade consistiu na preparação técnica relacionada ao \textit{framework} em javascript React Native, utilizado para desenvolver aplicativos móveis compatíveis com as plataformas Android e iOS.

\item[2.3. Estudos sobre UI/UX:] Embora o foco da aplicação seja a aprendizagem móvel, é indispensável conhecer os conceitos de Experiência de Usuários (UX) e Interface de Usuários (UI). Em vista disso, estudos direcionados ao assunto foram realizados a fim de possibilitar um melhor desenvolvimento da aplicação.


\end{description}

\item[3. Pesquisa sobre as principais funcionalidades de aplicativos de aprendizagem móvel:] Esta atividade consistiu no estudo de funcionalidades presentes em aplicativos de aprendizagem. Foram analisadas diversas plataformas de ensino móvel e decidiu-se explorar as funcionalidades dos principais.

% \item[4. Pesquisa sobre o algoritmo de geração das palavras cruzadas:] Esta atividade consistiu no estudo de um algoritmo eficiente que pudesse gerar palavras cruzadas em um tempo hábil para utilização do aplicativo móvel.

\item \textbf{4. Estudo do algoritmo de geração das palavras cruzadas:}
Esta atividade consistiu no estudo de um algoritmo eficiente que pudesse gerar palavras cruzadas em um tempo hábil para utilização do aplicativo móvel.
 

\begin{description}
\item[4.1. Força Bruta e Backtracking:] Detalhamento e explicação do conceito de Força Bruta e Backtracking para compreensão do algoritmo.

\item[4.2. Cadeias de Markov:] Definição do conceito de Cadeias de Markov no campo de Processos Estocásticos para estipulação do tempo de execução do algoritmo.

\item[4.3. Desenvolvimento do algoritmo:] Esclarecimento dos passos que definem o algoritmo.

\end{description}

\end{description}