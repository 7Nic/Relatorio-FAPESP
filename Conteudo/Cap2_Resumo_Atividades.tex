\chapter{Resumo das Atividades Desenvolvidas} \label{sec:resumo_ativ}
Nesta seção, são descritas sucintamente as principais atividades conduzidas pelo aluno no período de novembro/2019 a abril/2020 para alcançar os objetivos deste trabalho:

\begin{description}
\item[1. Estudo sobre usabilidade e acessibilidade:]
Esta atividade consistiu no estudo teórico dos conceitos de acessibilidade e usabilidade em contextos gerais da sociedade.

\item[2. Estudo sobre aprendizagem móvel:] Esta atividade consistiu no aprofundamento e estudo dos conceitos sobre aprendizagem móvel. Foram analisados os principais conceitos e características, bem como vantagens e limitações relacionadas à área. 

\begin{description}
    \item[2.1. Visão geral:]
    Esta atividade consistiu em adquirir uma visão geral a respeito da aprendizagem móvel (do que se trata, vantagens e desvantagens).
    
    \item[2.2. Estudo sobre as principais funcionalidades de aplicativos de aprendizagem móvel:]
    Esta atividade consistiu no estudo de funcionalidades presentes em aplicativos de aprendizagem móvel. Foram analisadas diversas plataformas de ensino móvel, sendo investigadas as funcionalidades dos principais deles.
\end{description}

\item[3. Estudo sobre ensino voltado para idosos:] Nesta atividade mostrou-se a importância da educação continuada como uma forma de gerar mudança na abordagem de processos educacionais.
Além disso, foram abordadas dificuldades do público idoso no aprendizagem bem como possíveis ações para lidar com elas.

\item[4. Artefatos e requisitos:]
A atividade foi composta pelo estudo das diretrizes de acessibilidade denominadas WCAG 2.1. Além disso, alguns requisitos foram estudados, os quais foram baseados no catálogo ReqML-Catalog \citep{soad2017reqml} e na linguagem de padrões MLearning-PL \citep{Fioravanti2017_plop}.

\begin{description}
    \item[4.1. Recomendações da WCAG 2.1:]
    Esta atividade consistiu no estudo das recomendações da WCAG (\textit{Web Content Accessibility Guidelines} -- Orientações para Acessibilidade em Conteúdos Web) para direcionar a construção do aplicativo de acordo com critérios de acessibilidade.
    
    \item[4.2. Requisitos:]
    A atividade consistiu no estudo dos requisitos elicitados tendo como base o catálogo ReqML-Catalog (Catálogo de Requisitos para Aplicações Educacionais Móveis), e a Linguagem de Padrões MLearning-PL com o fim de basear a elaboração e desenvolvimento do aplicativo.
\end{description}


\item[5. Levantamento e estudo de métodos e tecnologias para desenvolvimento de aplicações móveis:]
A atividade se compôs do levantamento de métodos para o desenvolvimento do aplicativo móvel. Iniciou-se explicando a metodologia do Scrum; a seguir foi descrito o \textit{framework} React Native utilizado no código da aplicação e conceitos importantes de UI/UX para a usabilidade de aplicativo. 

\begin{description}
\item[5.1. Scrum:] Esta atividade consistiu no estudo e entendimento do Scrum \citep{schwaber1997scrum}, metodologia ágil para gestão e planejamento de projetos de software. %, bem como no estudo sobre conceitos de requisitos funcionais, não funcionais e de domínio.

\item[5.2. React Native:] Esta atividade consistiu no estudo e entendimento do \textit{framework} em javascript, React Native \citep{RN}, utilizado para desenvolver aplicativos móveis compatíveis com as plataformas Android e iOS.

\item[5.3. UI/UX:] Para o desenvolvimento de aplicações para aprendizagem móvel, é fundamental que sejam considerados os conceitos de Experiência de Usuários (UX) e Interface de Usuários (UI). Em vista disso, estudos direcionados ao assunto foram realizados a fim de possibilitar um melhor desenvolvimento da aplicação.

\end{description}

\item[6. Projeto e Desenvolvimento da \textit{Crossword Learning}:]
Esta atividade consistiu na prototipação e desenvolvimento do aplicativo. Primeiro foi analisado o protótipo inicial, um \textit{mockup} composto pela interface do aplicativo. Além disso foi estudado um algoritmo que permitisse a construção de um tabuleiro com palavras cruzadas; e, por fim, o desenvolvimento de um MVP para a primeira validação.

\begin{description}
    \item[6.1. Protótipo inicial:]
    A atividade consistiu na análise do protótipo anteriormente construído com o objetivo de utilizar a pesquisa de validação efetuada.
    
    \item[6.2. Estudo do algoritmo de geração das palavras cruzadas:]
    Esta atividade consistiu no estudo de um algoritmo que pudesse criar um tabuleiro de palavras cruzadas com eficiência utilizando uma pequena lista de palavras fornecidas pelo usuário.
    
    \item[6.3. Desenvolvimento do \textit{MVP}:] 
    Esta atividade consistiu na construção de um Produto Mínimo Viável (\textit{Minimum Viable Product} -- MVP) a fim de realizar a validação preliminar de acessibilidade e usabilidade do aplicativo desenvolvido.
\end{description}

\end{description}