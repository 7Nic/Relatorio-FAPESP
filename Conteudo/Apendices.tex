\appendix

\chapter{Tabela de problemas encontrados na avaliação heurística}
\label{sec:tabela_av_heuristica}

\begingroup
\renewcommand{\arraystretch}{0.1} % Default value: 1
\begin{table}[H]
\centering
% \caption{\textit{Problemas encontrados na avaliação heurística}}
\centering
\footnotesize
\begin{tabular}{p{1.5cm} p{2cm} p{4cm} p{8.5cm}}
\toprule
\textbf{Heurística} & \textbf{Grau de severidade} & \textbf{Tela(s)} & \textbf{Descrição do problema}       
\\ \midrule
5
& 
2
&
Login
&
Primeira letra é sempre maiúscula, gerando erro ano tentar fazer o login.
\\ \midrule
7
& 
1
&
Login
&
Letras pequenas dificultam a leitura para usuários com problema de visão.
\\ \midrule
4
& 
2
&
Login
&
O logo do aplicativo está pequeno.
\\ \midrule

1
& 
2
&
Login
&
Não há \textit{feedback} para a ação efetuada
\\ \midrule
9
& 
2
&
Login
&
A mensagem de erro não é precisa, se os campos estiverem em branco, a mensagem indica email incorreto.
\\ \midrule
2
& 
2
&
Perfil
&
O ícone do nome é pouco usual.
\\ \midrule
2, 4
& 
1
&
Perfil
&
Se há apenas uma aula concluída, a escrita continua no plural.
\\ \midrule
4, 8
& 
1
&
Perfil
&
Ícone e texto de nome estão desalinhados do restante dos dados.
\\ \midrule
8
& 
1
&
Conteúdo textual
&
O texto apresentado não está justificado.
\\ \midrule
4
& 
1
&
Conteúdo textual
&
Seria interessante ter uma tamanho máximo e mínimo para a fonte do texto.
\\ \midrule
4
& 
2
&
Conteúdo em áudio
&
Há dois botões para controle de execução, usuários podem ficar confusos.
\\ \midrule
4
& 
3
&
Conteúdo em texto, Conteúdo em áudio e Conteúdo em vídeo
&
O botão de próxima tela deveria ficar em uma posição fixa.
\\ \midrule
4
& 
3
&
Conteúdo em áudio
&
Os botões de avançar e retroceder não tem finalidade.
\\ \midrule
4
& 
4
&
Conteúdo em vídeo
&
O vídeo continua sendo reproduzido após o botão de próxima tela ser clicado.
\\ \midrule
8
& 
3
&
Conteúdo em vídeo
&
Há dois botões para reprodução em tela cheia, usuários podem ficar confusos.
\\ \midrule
10
& 
3
&
Tutorial
&
Fica confuso se o que está sendo exibido é um tutorial ou se o jogo já iniciou. Seria interessante deixar explícito em todo o tempo de que se trata de um tutorial.
\\ \midrule
4
& 
2
&
Tutorial
&
A fonte utilizada não é a mesma da etapas anteriores.
\\ \midrule
6
& 
4
&
Palavra cruzada
&
O mecanismo de preenchimento é confuso: não há botões para apagar.
\\ \midrule
6
& 
4
&
Palavra cruzada
&
Letras pertencentes a palavras já preenchidas corretamente em "cruzamentos" precisam ser redigitadas.
\\ \midrule
3, 7
& 
4
&
Palavra cruzada
&
Quando o celular fica na horizontal, não é possível preencher as palavras cruzadas.
\\ \midrule
3
& 
4
&
Palavra cruzada
&
Não há botão para retornar, é necessário terminar o jogo para sair da tela de palavra cruzada.
\\ \midrule
7
& 
1
&
Palavra cruzada
&
As sombras dos botões de teclado virtual podem deixar a vista do usuário cansada.
\\ \bottomrule

\end{tabular}
\label{tab:avaliacaonielsen}
\end{table}
\endgroup