\chapter{Introdução}
\label{sec:introd}
A expectativa de vida no mundo subiu cerca de cinco anos no período de 2000 a 2015, de acordo com a \cite{world2016world}. Já no Brasil, no período de 1940 a 2016, a expectativa de vida foi elevada em mais de 30 anos \citep{expectativabrasileiros}. Ademais, segundo o \cite{demografico2010disponivel}, espera-se que a quantidade de pessoas com 60 anos ou mais quadruplique até 2060, atingindo um percentual de 26,8\% da população brasileira. 

Paralelamente às mudanças na expectativa de vida da população, podem ser presenciadas grandes transformações no âmbito das tecnologias de informação. O acesso à informação de maneira simples e rápida tem se tornado mais presente no cotidiano, possibilitado pela maior disseminação e fácil acesso à informação \citep{Grossi2014}. Favorecido por esse cenário, surgiram novas formas de lidar com as deficiências e limitações do ensino tradicional por meio de novas modalidades de ensino \citep{Behrens2005}. Particularmente, o avanço tecnológico permitiu miniaturizar, baratear e melhorar o desempenho de dispositivos eletrônicos, tornando-se muitas vezes superiores a computadores \citep{Zamfirache2013}. Isso permitiu uma nova modalidade de ensino: a aprendizagem móvel ou \textit{mobile learning} (\textit{m-learning}) \citep{Crompton2013, Keegan2005, Traxler2006, Wu2012}, a qual gera expectativas, pelo fato de beneficiar e flexibilizar o ensino e a aprendizagem.

Uma grande vantagem da flexibilização proporcionada pela aprendizagem móvel é a democratização do acesso à educação, incluindo-se, em especial, os aplicativos educacionais móveis para idosos. Todavia, é necessária atenção à acessibilidade e às propostas pedagógicas dessas aplicações, já que idosos podem apresentar comprometimento das capacidades mentais e físicas. Portanto, faz-se necessária a utilização de artefatos que foquem neste público como: diretrizes pedagógicas e de acessibilidade, padrões pedagógicos e catálogo de requisitos \citep{Oliveira2019_quali}. Concomitantemente devem ser levados em consideração aspectos de usabilidade, assim como em todo tipo de produto digital com interação humana. De acordo com \cite{zoemack_importance_usability2009}, a usabilidade se estabeleceu como um importante aspecto do design de um produto. Dessa forma, neste projeto, a finalidade é permitir que o usuário utilize o aplicativo focando no objetivo relacionado ao ensino, evitando perdas de tempo e esforço relativas a dificuldades de uso.

% Paragrafo sobre usabilidade aqui

Dessa forma, este projeto de iniciação científica tem o intuito de desenvolver uma aplicação educacional móvel, chamada \crossword, cujo objetivo é estimular o ensino e aprendizagem em idosos por meio de palavras cruzadas. O aplicativo deve apresentar uma interface intuitiva e de fácil uso, além da necessidade de se considerar aspectos pedagógicos, de acessibilidade e usabilidade em seu desenvolvimento. Além disso, o aplicativo pode ser voltado ao ensino de conhecimentos gerais, de diferentes áreas, tais como História, Geografia, Ciências, dentre outras \citep{oliveira2018crossword}. 

A fim de ser adaptável a perfis distintos de usuários, a aplicação possibilitará a escolha por níveis de dificuldades (fácil, médio ou difícil). Além disso, ao realizar as atividades propostas, o usuário passa a ganhar pontos, aumentando o seu nível e realizando atividades mais complexas. Ainda, devem ser disponibilizadas: (i) apresentação de conteúdos por vídeo, áudio e texto; (ii) compartilhamento de resultados; (iii) monitoramento de nível por \textit{ranking} pessoal \citep{oliveira2018crossword}. 

Nesta seção foram apresentados o contexto, a motivação e objetivo principal deste trabalho de iniciação científica. Na Seção \ref{sec:resumo_ativ} é apresentado um resumo das atividades realizadas, as quais são detalhadas na Seção \ref{sec:ativ_desenvolvidas}. Por fim, na Seção \ref{sec:atividades_futuras} são listadas as atividades previstas em outros projetos que possam complementar o atual.
% as futuras atividades referentes ao projeto e, na Seção \ref{sec:conclusao} são feitas as considerações finais.