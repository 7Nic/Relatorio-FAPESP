\documentclass[10pt,a4paper,final,brazil,oneside,pdftex] {article}
\usepackage{graphicx} %tabelas
\usepackage{siunitx} %tabelas
\usepackage{booktabs,tabularx}
\usepackage[brazil]{babel}
\usepackage{amsfonts}
\usepackage[pdftex]{graphicx}
\usepackage[utf8]{inputenc} % permite acentuação correta
\usepackage{times}    % ??
\usepackage{setspace} % ??
\usepackage{xspace}   % ??
\usepackage{float}
\usepackage{color}
\usepackage{colortbl}
\usepackage{verbatim}
\usepackage{textfit}
\usepackage{enumerate}
\usepackage{multirow}
\usepackage{alltt}
\usepackage[T1]{fontenc}
\usepackage[top=10mm,bottom=20mm,left=20mm,right=20mm,pdftex,includeheadfoot]{geometry}
\usepackage{ae}
\usepackage{fancyhdr}
\usepackage{fancybox}
\usepackage{multicol}
\usepackage{listings} % formata código-fonte inserido no documento
\usepackage{subfigure}
\usepackage{placeins} % auxilia no posicionamento de figuras
\usepackage{url}
\usepackage[%numbers,
            authoryear,
            sort&compress,]{natbib}

\usepackage[pdftex]{hyperref}  % gera os hiperlinks no pdf
 \hypersetup{colorlinks=true,debug=false,
   linkcolor=black,%%%cor do tableofcontents,\ref,\footnote,etc
   citecolor=black, %%% cor do \cite
   urlcolor=black, %%% cor do \url e \href
   pdftitle={Relatório Final},
   pdfauthor={Autor},
   pdfsubject={Titulo}}


\newcommand{\reqmlcatalog}{\textit{ReqML-Catalog}\xspace}
\newcommand{\mlearningpl}{\textit{MLearning-PL}\xspace}
\newcommand{\crossword}{\textit{Crossword Learning}\xspace}

\pdfcompresslevel=9
\DeclareGraphicsExtensions{.png,.jpg,.pdf,.mps} % Para versao PDF

%*****************************  Definições de estilo   ************************

\let\chapter=\section
\let\section=\subsection
\let\subsection=\subsubsection

\bibliographystyle{icmc2}

\widowpenalty=10000
\clubpenalty=10000
\exhyphenpenalty=10000


% comandos para definir espaço entre figuras, captions e texto
\setlength{\abovecaptionskip}{0.2cm}
\setlength\textfloatsep{12pt}
\setlength\floatsep{12pt}

\headheight 17.1pt

\setcounter{secnumdepth} {3} % Ajusta o numero de capítulos para 3 níveis
\setcounter{tocdepth} {3} % Faz com que os 3 níveis de capítulos apareçam no índice
\definecolor{gray}{rgb}{0.7,0.7,0.7} % definição de cor cinza

\usepackage{xspace}


%*********************   Capa  ******************************************

\begin{document}

\pagestyle{empty}
\thispagestyle{empty}

\begin{titlepage}

\begin{center}
	\begin{figure}[!ht]
	\centering
	\includegraphics[scale=0.5]{Figuras/logo.png}
	\end{figure}
	\vspace{60pt}
	\begin{minipage}[c]{13.5cm}
			\begin{center}
			\vspace{1.3cm} {\Large\sf Relatório Parcial de Iniciação Científica}\\
			\vspace{0.3cm} {\Large\sf FAPESP}\\
			\vspace{5cm}
			\textbf{\Large\sf\textit{Crossword Learning: projeto e desenvolvimento de um
aplicativo educacional móvel com foco em usuários
idosos}}\\
			\vspace{1.3cm}
			\end{center}
		\end{minipage}
\vfill
\begin{minipage}[c]{13.5cm}
	\begin{center}
		\sffamily\textbf{Bolsista:} Gabriel Santos Nicolau\\
		\sffamily\textbf{Orientadora:} Ellen Francine Barbosa\\
		\sffamily\textbf{Período:} Setembro/2019 a Agosto/2020\\
	\end{center}
\end{minipage}
\end{center}
\cleardoublepage
\end{titlepage}

\clearpage{\pagestyle{empty}\cleardoublepage}


%**************************  Índice ***********************************************

\pagestyle{plain}
\thispagestyle{plain}
\renewcommand{\thepage}{\roman{page}}
\setcounter{page}{1}

\tableofcontents
%\cleardoublepage
%\listoffigures
%\cleardoublepage
%\listoftables
\clearpage{\pagestyle{empty}\cleardoublepage}

\newpage
\onehalfspace

\pagestyle{plain}   % define estilo de página
\fancyfoot[C]{\thepage}  % insere numero da pagina no rodapé
\fancyhead[R]{\small\leftmark}
\lhead{}            % define o cabeçalho do lado direito como sendo vazio
\chead{}					  % define o cabeçalho do centro como sendo vazio
%\let\cite=\citep    % define o formato/tipo de citações.

%*******************************************************************************
\newpage
\renewcommand{\thepage}{\arabic{page}}
\setcounter{page}{1}

%\renewcommand{\chaptermark}[1]{\markboth{\textit{ \chaptername \ \thechapter.\ #1}}{}}
\renewcommand{\sectionmark}[1]{\markright{\thesection.\ #1}}

%**********************************************************************************************
%********************** DOCUMENTO ************************************************************

\chapter{Introdução}
\label{sec:introd}
A expectativa de vida no mundo subiu cerca de cinco anos no período de 2000 a 2015, de acordo com a \cite{world2016world}. Já no Brasil, no período de 1940 a 2016, a expectativa de vida foi elevada em mais de 30 anos \citep{expectativabrasileiros}. Ademais, segundo o \cite{demografico2010disponivel}, espera-se que a quantidade de pessoas com 60 anos ou mais quadruplique até 2060, atingindo um percentual de 26,8\% da população brasileira. 

Paralelamente às mudanças na expectativa de vida da população, podem ser presenciadas grandes transformações no âmbito das tecnologias de informação. O acesso à informação de maneira simples e rápida tem se tornado mais presente no cotidiano, possibilitado pela maior disseminação e fácil acesso à informação \citep{Grossi2014}. Favorecido por esse cenário, surgiram novas formas de lidar com as deficiências e limitações do ensino tradicional por meio de novas modalidades de ensino \citep{Behrens2005}. Particularmente, o avanço tecnológico permitiu miniaturizar, baratear e melhorar o desempenho de dispositivos eletrônicos, tornando-se muitas vezes superiores a computadores \citep{Zamfirache2013}. Isso permitiu uma nova modalidade de ensino: a aprendizagem móvel ou \textit{mobile learning} (\textit{m-learning}) \citep{Crompton2013, Keegan2005, Traxler2006, Wu2012}, a qual gera expectativas, pelo fato de beneficiar e flexibilizar o ensino e a aprendizagem.

Uma grande vantagem da flexibilização proporcionada pela aprendizagem móvel é a democratização do acesso à educação, incluindo-se, em especial, os aplicativos educacionais móveis para idosos. Todavia, é necessária atenção à acessibilidade e às propostas pedagógicas dessas aplicações, já que idosos podem apresentar comprometimento das capacidades mentais e físicas. Portanto, faz-se necessária a utilização de artefatos que foquem neste público como: diretrizes pedagógicas, padrões pedagógicos e catálogo de requisitos \citep{Oliveira2019_quali}.

Dessa forma, este projeto de iniciação científica tem o intuito de desenvolver uma aplicação educacional móvel, chamada \crossword, cujo objetivo é estimular o ensino e aprendizagem em idosos por meio de palavras-cruzadas. O aplicativo deve apresentar uma interface intuitiva e de fácil uso, além da necessidade de se considerar aspectos de acessibilidade em seu desenvolvimento. Além disso, o aplicativo pode ser voltado ao ensino de conhecimentos gerais, de diferentes áreas, tais como História, Geografia, Ciências, dentre outras \citep{oliveira2018crossword}. 

A fim de ser adaptável a perfis distintos de usuários, a aplicação possibilitará a escolha por níveis de dificuldades (fácil, médio ou difícil). Além disso, ao realizar as atividades propostas, o usuário passa a ganhar pontos, aumentando o seu nível e realizando atividades mais complexas. Ainda, devem ser disponibilizadas: (i) apresentação de conteúdos por vídeo, áudio e texto; (ii) compartilhamento de resultados; (iii) monitoramento de nível por \textit{ranking} pessoal \citep{oliveira2018crossword}. 

Nesta seção foram apresentados o contexto, a motivação e objetivo principal deste trabalho de iniciação científica. Na Seção \ref{sec:resumo_ativ} é apresentado um resumo das atividades realizadas, as quais são detalhadas na Seção \ref{sec:ativ_desenvolvidas}. Por fim, na Seção \ref{sec:atividades_futuras} são listadas as atividades em andamento/previstas. % as futuras atividades referentes ao projeto e, na Seção \ref{sec:conclusao} são feitas as considerações finais.
\chapter{Resumo das Atividades Desenvolvidas} \label{sec:resumo_ativ}
Nesta seção, são descritas sucintamente as principais atividades conduzidas pelo aluno no período de novembro/2019 a abril/2020 para alcançar os objetivos deste trabalho:

\begin{description}
\item[1. Estudo sobre usabilidade e acessibilidade:]
Esta atividade consistiu no estudo teórico dos conceitos de acessibilidade e usabilidade em contextos gerais da sociedade.

\item[2. Estudo sobre aprendizagem móvel:] Esta atividade consistiu no aprofundamento e estudo dos conceitos sobre aprendizagem móvel. Foram analisados os principais conceitos e características, bem como vantagens e limitações relacionadas à área. 

\begin{description}
    \item[2.1. Visão geral:]
    Esta atividade consistiu em adquirir uma visão geral a respeito da aprendizagem móvel (do que se trata, vantagens e desvantagens).
    
    \item[2.2. Estudo sobre as principais funcionalidades de aplicativos de aprendizagem móvel:]
    Esta atividade consistiu no estudo de funcionalidades presentes em aplicativos de aprendizagem móvel. Foram analisadas diversas plataformas de ensino móvel, sendo investigadas as funcionalidades dos principais deles.
\end{description}

\item[3. Estudo sobre ensino voltado para idosos:]
A atividade foi composta explicação da necessidade de mudança da abordagem na educação com o uso da educação continuada. Além disso foram abordadas dificuldades do público idoso no aprendizagem bem como possíveis ações para lidar com elas.

\item[4. Estudo sobre artefatos:]
\hfill

\begin{description}
    \item[4.1. Recomendações da WCAG 2.1:]
    Esta atividade consistiu no estudo das recomendações da WCAG (\textit{Web Content Accessibility Guidelines} -- Orientações para Acessibilidade em Conteúdos Web) para direcionar a construção do aplicativo de acordo com critérios de acessibilidade.
    
    \item[4.2. ReqML Catalog -- Catálogo de Requisitos \citep{soad2017reqml}:]
    A atividade consistiu na análise do referido catálogo de requisitos a fim de basear a elaboração e desenvolvimento do aplicativo.
\end{description}


\item[5. Levantamento e estudo de métodos e tecnologias para desenvolvimento de aplicações móveis:]
\hfill

\begin{description}
\item[5.1. Scrum:] Esta atividade consistiu no estudo e entendimento do Scrum, metodologia ágil para gestão e planejamento de projetos de software. %, bem como no estudo sobre conceitos de requisitos funcionais, não funcionais e de domínio.

\item[5.2. React Native:] Esta atividade consistiu no estudo e entendimento do \textit{framework} em javascript React Native, utilizado para desenvolver aplicativos móveis compatíveis com as plataformas Android e iOS.

\item[5.3. UI/UX:] Para o desenvolvimento de aplicações para aprendizagem móvel, é fundamental que sejam considerados os conceitos de Experiência de Usuários (UX) e Interface de Usuários (UI). Em vista disso, estudos direcionados ao assunto foram realizados a fim de possibilitar um melhor desenvolvimento da aplicação.

\end{description}

\item[6. Projeto e Desenvolvimento da \textit{Crossword Learning}:]
\hfill

\begin{description}
    \item[6.1. Protótipo inicial:]
    A atividade consistiu na análise do protótipo anteriormente construído com o objetivo de utilizar a pesquisa de validação efetuada.
    
    \item[6.2. Estudo do algoritmo de geração das palavras cruzadas:]
    Esta atividade consistiu no estudo de um algoritmo que pudesse criar um tabuleiro de palavras cruzadas com eficiência utilizando uma pequena lista de palavras fornecidas pelo usuário.
    
    \item[6.3. Desenvolvimento do \textit{MVP}:] 
    Esta atividade consistiu na construção de um Produto Mínimo Viável (\textit{Minimum Viable Product} -- MVP) a fim de realizar a validação preliminar de acessibilidade e usabilidade do aplicativo desenvolvido.
\end{description}

\end{description}
\chapter{Atividades Desenvolvidas}\label{sec:ativ_desenvolvidas}
Nesta seção são descritas, em detalhes, as principais atividades realizadas para a condução deste projeto.

%Ênfase foi dada ao estudo teórico que pudesse embasar o desenvolvimento do aplicativo em si; conceitos como aprendizagem móvel, acessibilidade e usabilidade foram abordados. Além desses, é apresentado o estudo das tecnologias utilizadas para a construção do código do aplicativo.

\section{Estudo sobre usabilidade e acessibilidade}\label{sec:estudos_usab_acess} 

%Como primeiro ponto, foram abordados os aspectos relacionados ao conceito de usabilidade bem como a definição de acessibilidade. 

Segundo a Lei 10.098, de 19 de dezembro de 2000 da Legislação Brasileira\footnote{\url{https://www2.camara.leg.br/legin/fed/lei/2000/lei-10098-19-dezembro-2000-377651-publicacaooriginal-1-pl.html}}, a acessibilidade é a possibilidade e condição de alcance para utilização, com segurança e autonomia, dos
espaços, mobiliários e equipamentos urbanos, das edificações, dos transportes e dos sistemas e meios de comunicação por pessoa portadora de deficiência ou com mobilidade reduzida. 

De modo complementar, \cite{torres2002acessibilidade} diz que acessibilidade é um processo dinâmico, associado não só ao desenvolvimento tecnológico, mas principalmente ao desenvolvimento da sociedade. Em geral, está associada a pessoas com deficiências físicas ou psicológicas, idosos ou grupos excluídos; este último mais bem relacionado com o termo \textit{inclusão}. 

Embora a acessibilidade não se restrinja ao meio digital, é essencial que esteja presente no mesmo. Corroborando com essa ideia, \cite{leew3c} afirma a importância da Web ser utilizável por qualquer um, independente de capacidades individuais ou deficiências. No cenário atual, isso também se aplica ao uso de aplicativos em dispositivos móveis.

Apesar da elicitação da acessibilidade no campo jurídico, social e meio digital, faz-se necessário um foco adicional ao campo educacional. De acordo com \cite{Bine2018DigitalIT}, a educação inclusiva se tornou uma das questões mais desafiadoras dos sistemas educacionais atuais. Isso ocorre em parte pela falta de preparo de professores e deficiễncia de material.

Atrelada à acessibilidade deve estar a usabilidade que, segundo \cite{nielsenPrioritizingWebUsability},
é o atributo relacionado a quão fácil é usar algo. Mais especificamente, refere-se a quão rápido pessoas podem aprender a usar algo, quão eficientes são enquanto o utilizam, quão memorável é, quão propenso a erros e o quanto as pessoas gostam de utilizar; deve-se tornar a utilização fluida. É interessante destacar que existe uma frase entre desenvolvedores que diz: ''não faça o usuário pensar`` baseada no livro \textit{Don't make me think} \citep{steveDont2005}. O objetivo é tornar o processo intuitivo o suficiente a ponto de ser considerado natural.
Ainda, é essencial analisar fatores como capacidade de aprendizagem, a fim de que não seja necessário um aprendizado extra por parte do usuário para utilizar o sistema em questão.

No contexto em que se insere este trabalho isso se torna ainda mais essencial, uma vez que o público idoso enfrenta vários problemas relacionados à usabilidade em \textit{smartphones} \citep{dificuldadesIdosos} e principalmente à acessibilidade. %Isso ocorre, principalmente, pois investidores da área de tecnologia consideram um grau de incerteza no que tange oportunidades de negócios relacionadas à área quando se trata de usuários idosos \citep{NBCelderly}.

%Pelo exposto acima, foi decidido seguir recomendações de usabilidade e acessibilidade no desenvolvimento do projeto. 

Ressalta-se que o objetivo principal do projeto é auxiliar o processo de ensino-aprendizagem de usuários idosos. Portanto, é importante que recomendações de usabilidade e acessibilidade sejam consideradas, de modo a possibilitar que o usuário foque no conteúdo do aplicativo, e não desvie sua atenção por dificuldades de uso.

\section{Estudo sobre aprendizagem móvel}\label{sec:estudos_ap_movel} 
Outra atividade realizada foi o estudo do conceito de aprendizagem móvel, bem como uma pesquisa sobre as principais funcionalidades presentes em aplicativos relacionados ao ensino de palavras-cruzadas, tendo como público-alvo o usuário idoso.

\subsection{Visão geral}
Existem várias definições de \textit{m-learning}. %Assim, ao longo dos anos, pesquisadores se dispuseram a estudar o assunto e tentaram achar uma definição. 
De acordo com \cite{Quinn2000}, \textit{m-learning} é um modelo de aprendizagem eletrônica (\textit{e-learning}) que utiliza equipamentos computadorizados: \textit{Palmtops}, dispositivos que rodam \textit{Windows Embedded Compact} e até um telefone celular.
Em 2011, \cite{hwang2011research} argumentaram que uma definição amplamente aceita de \textit{m-learning} é simplesmente ``usar tecnologias móveis para facilitar o aprendizado''. Dentre outras definições, a adotada para esse projeto é a de \cite{traxler2005defining}: qualquer fornecimento educacional onde a tecnologia dominante é portátil ou dispositivos \textit{palmtop}.

Todavia, independentemente da definição considerada, torna-se necessário analisar as vantagens e desvantagens de se utilizar dispositivos móveis no processo de ensino-aprendizagem. De acordo com \cite{RICHAMEHTA2016} destacam-se como vantagens: 

\begin{itemize}
    \item Tablets com anotações e e-books são mais leves e menos volumosos que mochilas cheias de papéis, livros e até laptops;
    \item É mais fácil acomodar dispositivos móveis em uma sala se comparado à computadores de mesa;
    \item Dispositivos móveis podem ser usados em qualquer lugar e em qualquer momento, como em trens, em casa, em hotéis; e isto tem um valor inestimável para a educação \citep{CarmaMaia2008}.
\end{itemize}

Porém, de acordo com \cite{RICHAMEHTA2016}, as seguintes desvantagens devem ser destacadas: 

\begin{itemize}
    \item Celulares pequenos limitam-se a quantidade e tipo de informação que pode ser exibida;
    \item Baterias precisam ser recarregadas regularmente, e dados podem ser perdidos caso não se faça o carregamento correto;
    \item É difícil usar gráficos que possuem movimento, especialmente em celulares pequenos.
\end{itemize}

Dessa forma, devido ao processo de envelhecimento, limitações e desafios podem ser potencializadas no caso de usuários idosos. Além disso, é necessário que as aplicações educacionais móveis levem em consideração propostas pedagógicas adequadas e específicas para esse público. Logo, é indispensável que o desenvolvimento dessas aplicações seja realizado de maneira clara e objetiva, possibilitando melhor aprendizagem por parte do idoso \citep{giubilei1993pedagogia}.

\subsection{Estudo sobre as principais funcionalidades de aplicativos de aprendizagem móvel}

Diversos são os aplicativos que visam apoiar o processo de ensino-aprendizagem do usuário, seja por meio de jogos, vídeos ou outros materiais. 

Dessa maneira, alguns aplicativos (para diferentes públicos) foram analisados, a fim de verificar suas funcionalidades e propostas de aprendizagem e acessibilidade, de modo que essas pudessem ser adaptadas ou utilizadas para aplicações com foco no idoso.

\begin{description}
% Ensina o que? Qual o publico alvo? Pessoas de quantos anos?

\item[Engaging congress]\footnote{\url{https://play.google.com/store/apps/details?id=com.iu.engagingcongress&hl=en}, \url{https://apps.apple.com/us/app/engaging-congress/id1309161238?ls=1}} \hfill \\
\textit{Enganging congress} (\autoref{fig:EngCong}) é um jogo interativo que visa explorar os princípios básicos de um governo representativo. Escolhe-se um tema e é exibido um vídeo. Jogos são incluídos no processo baseados no tema em questão. É importante destacar a atenção dos criadores em fazer o usuário compreender a tarefa; a todo momento é possível clicar no botão de dúvida. As principais funcionalidades são: (i) vídeos educativos sobre o tema; (ii) perguntas relacionadas ao conteúdo passado; (iii) nota final após a conclusão dos exercícios; (iv) botões de dúvidas sempre presentes.

\begin{figure}[ht!]
\centering
    \caption{Telas do aplicativo \textit{Engaging Congress}}
    \label{fig:EngCong}
    \includegraphics[width=0.9\textwidth]{Figuras/engagingcongress.png}
    
    Fonte: Capturas de tela do aplicativo\footnote{\url{https://play.google.com/store/apps/details?id=com.iu.engagingcongress&hl=en}, \url{https://apps.apple.com/us/app/engaging-congress/id1309161238?ls=1}}
\end{figure}

\item[Play PBS KIDS Games]\footnote{\url{https://apps.apple.com/us/app/pbs-kids-games/id1050773989}, \url{https://play.google.com/store/apps/details?id=org.pbskids.gamesapp&hl=pt_BR}} \hfill \\
O aplicativo \textit{Play PBS KIDS Games} (\autoref{fig:pbs}) visa a promoção da educação para crianças na fase de alfabetização (2 a 8 anos), contendo mais de 100 mini-jogos voltados para tal. As crianças são encorajadas a resolver desafios e aprimorar suas habilidades em ciências, matemática, letras e criatividade. O objetivo é impactar positivamente a vida de crianças por meio de mídia baseada em um currículo onde quer que elas estejam. O aplicativo ganhou duas premiações no ano de 2017, a saber Melhor aplicativo de jogos para Pré-Escola (\textit{Kidscreen Award Winner}) e o \textit{Parents' Choice Recommended Mobile App}. Os principais recursos são: (i) funcionamento offline; (ii) possibilidade de gerenciar a quantidade de memória que será consumida; (iii) obtenção de detalhes sobre desenhos da TV PBS Kids, como idade recomendada e objetivos de aprendizado para as crianças.

\begin{figure}[H]
\centering
    \caption{Telas do aplicativo \textit{Play PBS KIDS Games}}
    \label{fig:pbs}
    \includegraphics[width=0.9\textwidth]{Figuras/pbsKids.jpg}
    
    Fonte: Capturas de tela do aplicativo\footnote{\url{https://apps.apple.com/us/app/pbs-kids-games/id1050773989}, \url{https://play.google.com/store/apps/details?id=org.pbskids.gamesapp&hl=pt_BR}}
\end{figure}

\item[Human Anatomy Atlas]\footnote{\url{https://apps.apple.com/br/app/human-anatomy-atlas-2020/id1117998129}, \url{https://play.google.com/store/apps/details?id=com.argosy.vbandroid&hl=pt_BR}} \hfill \\
O \textit{Human Anatomy Atlas} (\autoref{fig:humAtlas}) é um aplicativo criado por um time de especialistas em visualização biomédica. É direcionado ao ensino da anatomia do corpo humano com o foco em estudantes e professores, apesar de também ser utilizado em hospitais. Seus modelos em 3D promovem uma fidelidade às estruturas humanas reais, sendo possível rotacionar e dissecar órgãos e partes do corpo. Possui recursos como : (i) interatividade com estruturas 3D; (ii) mais de 1000 questões para testes em assuntos; (iii) visualização de anatomias complexas em realidade aumentada; (iv) disponibilidade em 7 idiomas.

\begin{figure}[ht!]
\centering
    \caption{Telas do aplicativo \textit{Human Anatomy Atlas}}
    \label{fig:humAtlas}
    \includegraphics[width=0.9\textwidth]{Figuras/humanAtlas.png}
    
    Fonte: \cite{HumanAnatomyAtlas}
\end{figure}

\item[Bini Super ABC]\footnote{\url{https://apps.apple.com/br/app/bini-abc-alfabeto-crianças-app/id1397966958}, \url{https://play.google.com/store/apps/details?id=com.binibambini.abc&hl=pt_BR}} \hfill \\
Bini Super ABC (\autoref{fig:biniABC}) é um aplicativo voltado para crianças na faixa de 3 a 5 anos na fase de instrução educacional. O aplicativo promove o aprendizado das letras do alfabeto com diversos jogos infantis. O objetivo é tornar o ensino interessante e empolgante com desenhos coloridos, personagens e efeitos sonoros. A aplicação possui as funcionalidades de: (i) aprendizado das letras por sons; (ii) reforço do material aprendido; (iii) controle de responsáveis para os jogos, entre outras.

\begin{figure}[H]
\centering
    \caption{Telas do aplicativo \textit{Bini Super ABC}}
    \label{fig:biniABC}
    \includegraphics[width=0.9\textwidth]{Figuras/biniabc.png}
    
    Fonte: Capturas de tela do aplicativo\footnote{\url{https://apps.apple.com/br/app/bini-abc-alfabeto-crianças-app/id1397966958}, \url{https://play.google.com/store/apps/details?id=com.binibambini.abc&hl=pt_BR}}
\end{figure}

\item[Crossword Puzzle Free]\footnote{\url{https://play.google.com/store/apps/details?id=mobi.redstonegames.crossword.en&hl=pt_BR}} \hfill \\
Crossword Puzzle Free (\autoref{fig:crossFree}) é um aplicativo de palavras cruzadas. É de fácil  uso por não exigir conhecimentos específicos do usuário. Possui quatro níveis de dificuldade a partir dos quais é possível buscar o desafio na medida certa além de motivar o usuário a continuar utilizando o aplicativo por não ser nem muito fácil nem muito difícil. As principais funcionalidades são: (i) teclado do próprio aplicativo; (ii) dicas aparecem acima do teclado; e (iii) botão de dica para a palavra, entre outras.


\begin{figure}[H]
\centering
    \caption{Telas do aplicativo \textit{Crossword Puzzle Free}}
    \label{fig:crossFree}
    \includegraphics[width=0.9\textwidth]{Figuras/crosswordPuzzleFree.jpg}
    
    Fonte: Capturas de tela do aplicativo\footnote{\url{https://play.google.com/store/apps/details?id=mobi.redstonegames.crossword.en&hl=pt_BR}}
\end{figure}

\item[Crossword Quiz]\footnote{\url{https://play.google.com/store/apps/details?id=com.randomlogicgames.crossword&hl=pt_BR}} \hfill \\
Crossword Quiz (\autoref{fig:crossQuiz}) é outro aplicativo de palavras cruzadas, porém com uma abordagem mais facilitada pelo fato de não se disponibilizarem todas as letras do teclado, tornando a resposta mais fácil para o usuário. Além disso, o aplicativo não conta apenas com dicas escritas, mas inclui imagens. As funcionalidade que se destacam são: (i) não mostrar todo o teclado, apenas algumas letras; (ii) mostra imagens como dicas, além de textos; e (iii) tutorial inicial intuitivo e completo.

\begin{figure}[H]
\centering
    \caption{Telas do aplicativo \textit{Crossword Quiz}}
    \label{fig:crossQuiz}
    \includegraphics[width=0.9\textwidth]{Figuras/crosswordQuiz.jpg}
    
    Fonte: Capturas de tela do aplicativo\footnote{\url{https://play.google.com/store/apps/details?id=com.randomlogicgames.crossword&hl=pt_BR}}
\end{figure}

% inicio n deixe a vovó cair

\item[Não deixe a vovó cair]\footnote{\url{https://play.google.com/store/apps/details?id=com.GazGames.SalveVovo&hl=pt_BR}} \hfill \\
Não deixe a vovó cair (\autoref{fig:nDeixeVovoCair}) é um aplicativo que visa reduzir os riscos do ambiente domiciliar. Produzido pelo Centro de Telessaúde do Hospital das Clínicas – UFMG (Universidade Federal de Minas Gerais) e pela Rede de Teleassistência de Minas Gerais, possui quatro níveis que representam áreas de uma casa: Banheiro, Quarto, Cozinha e Sala. Além disso, conta com a ajuda do cão Neca, que representa o cachorro da dona da casa, e guia o usuário nos passos do jogo.
\begin{figure}[H]
\centering
    \caption{Telas do aplicativo \textit{Crossword Quiz}}
    \label{fig:nDeixeVovoCair}
    \includegraphics[width=0.9\textwidth]{Figuras/nDeixeVovoCair.jpg}
    
    Fonte: Capturas de tela do aplicativo\footnote{\url{https://play.google.com/store/apps/details?id=com.GazGames.SalveVovo&hl=pt_BR}}
\end{figure}

% fim
\end{description}

Embora a área de atuação da maioria dos aplicativos considerados não seja relacionada especificamente ao público idoso, a análise conduzida foi importante visto que tais aplicativos tratarem de recursos educacionais voltados ao ensino. Além disso, os aplicativos voltados para crianças foram de grande utilidade na exemplificação de como fazer um produto com interface simples, intuitiva, fácil uso e com uma curva de aprendizado pequena. 

Na Tabela \ref{tab:funcionalidade} são listadas as funcionalidades principais que serviram como base para o desenvolvimento da \textit{Crossword Learning}.

\begin{table}[!ht]
\centering
\caption{\textit{Principais funcionalidades das aplicações educacionais}}
\centering
\footnotesize
\begin{tabular}{p{6cm} p{10cm}}
\toprule
\textbf{Aplicativo} & \textbf{Funcionalidades}                                              \\ \midrule

Engaging congress   & \begin{tabular}[c]{@{}l@{}}Vídeos educativos\end{tabular}                                                                 \\ \midrule
Play PBS KIDS Games & \begin{tabular}[c]{@{}l@{}}Funcionamento offline\end{tabular}      \\ \midrule

Human Anatomy Atlas & \begin{tabular}[c]{@{}l@{}}Sete idiomas disponíveis\end{tabular}    \\ \midrule

Bini Super ABC & \begin{tabular}[c]{@{}l@{}}Aprendizado das letras por sons\end{tabular}                \\ \midrule

Crossword Puzzle Free & \begin{tabular}[c]{@{}l@{}}Teclado não nativo, utiliza um personalizado\\ Dica acima do teclado\\ Botão de dica para a palavra\end{tabular}                \\ \midrule
Crossword Quiz & \begin{tabular}[c]{@{}l@{}}Não mostra o teclado todo, apenas algumas letras\\  Tutorial inicial mostrando todos os passos antes da primeira partida\end{tabular}                \\ \midrule
Não deixe a vovó cair & \begin{tabular}[c]{@{}l@{}}Tutorial inicial ensinando o uso do aplicativo \\ Interface intuitiva e de fácil compreensão \end{tabular}                \\ \midrule
\end{tabular}
\label{tab:funcionalidade}
Fonte: Elaborada pelo autor
\end{table}


\section{Estudo sobre ensino voltado para idosos}

%Um dos objetivos do projeto é desenvolver uma aplicação educacional voltada para o público idoso. Portanto, faz-se necessário analisar o contexto relacionado ao ensino voltado ao público idoso, a fim de auxiliar no desenvolvimento do projeto. 
Como já discutido na seção \ref{sec:introd}, a expectativa de vida no mundo subiu e, em países em desenvolvimento como o Brasil, esse cenário é intensificado com prospecções de quadruplicação de pessoas com 60 anos ou mais \citep{demografico2010disponivel}. Dessa maneira, surge um novo desafio: adequar as práticas educacionais a esse público. De acordo com \cite{rethinkingTeacherEducation}, a situação envolve aspectos que não podem ser trabalhados sem a reconstrução da atual abordagem na educação. É necessária a mudança da direção de pensamento sobre o público idoso aliada à construção de métodos que possam auxiliá-los ou enriquecer pedagogicamente as ofertas de ensino.

Dentre as possibilidades de abordagem, há a educação continuada ou educação permanente, a qual objetiva capacitar um indivíduo após o período escolar ou ao longo da vida da pessoa. Assim, aplicada à área Gerontológica esse tipo de educação pode ser utilizada em iniciativas educacionais \citep{neri2001palavras}. 

\cite{zemke198430}, citam princípios que se relacionam à aprendizagem de adultos, os quais podem ser aplicados ao público idoso. Há a divisão em três principais questões: (i) motivações para aprender; (ii) desenho curricular; e (iii) sala de aula. Há uma série de afirmações feitas pelos autores (Tabela \ref{tab:questoesAprIdosos}), que podem ser aplicadas ao ensino de idosos.

\begin{table}[!ht]
\centering
\caption{\textit{Questões que podem ser consideradas para o ensino-aprendizagem de idosos}}
\centering
\footnotesize
\begin{tabular}{p{5cm} p{5cm} p{5cm}}
\toprule
\textbf{Quanto às
motivações para aprender} & \textbf{Quanto ao
desenho curricular} & \textbf{Quanto à
sala de aula}                                   
\\ \midrule

Eventos específicos de mudanças de
vida (casamento, divórcio, aposentadoria, dentre outros) fazem com que
os adultos busquem experiências de
aprendizagem.
& 
Alunos adultos e idosos tendem a
preferir cursos práticos a cursos teóricos e conceituais.
&
Aulas expositivas e longas aumentam a taxa de irritação.
\\ \midrule

Quanto maior o número de eventos
de mudanças de vida, mais probabilidade de buscar esta experiência.
& 
Informações conflitantes àquelas
consideradas verdadeiras pelos adultos e idosos são integradas de maneira mais lenta.
&
Idosos trazem uma grande experiência de vida para a sala de aula e
esta vantagem deve ser reconhecida,
extraída e usada.
\\ \midrule

Os adultos (e idosos) acreditam que
terão uso do conhecimento ou habilidade adquirida, de maneira que
buscam aplicar e seguir aquilo que
aprenderam.
& 
Tarefas de passos rápidos, complexas e incomuns interferem na aprendizagem e entendimento.
&
Novos conhecimentos devem ser integrados aos conhecimentos prévios
do aluno.
\\ \midrule

O aumento ou manutenção da autoestima e prazer também são consideradas motivações para o ato de
aprender.
& 
Por serem mais lentos em algumas
tarefas de aprendizagem psicomotoras procuram não realizar tentativas e evitam errar, tendendo a uma
maior precisão nas tarefas.
&
O instrutor deve saber equilibrar os
conteúdos com as experiências relevantes dos alunos e o tempo de aula.
\\ \midrule


& 
É necessário que o desenho curricular seja baseado em conceitos e
ideias que estejam em concordância
com os educandos.
&
Integração de novos conhecimentos
e habilidades requerem um maior
tempo de transição e esforços.
\\ \midrule
\end{tabular}
\label{tab:questoesAprIdosos}

Fonte: \cite{zemke198430}
\end{table}

Para uma compreensão mais abrangente de todo o contexto de ensino voltado ao idoso, é importante á análise conjunta entre as questões apontadas e as principais dificuldades enfrentadas pelo público em questão \citep{euromed}:

\begin{itemize}
    \item Problemas em absorver novas informações;
    \item Crescente dificuldade de compreender frases complexas de acordo com a idade;
    \item Esforço para entender inferências;
    \item Dificuldades de audição;
    \item Problemas de coordenação motora;
    \item Perda crescente de visão.
\end{itemize}

Tendo em vista tais dificuldades, faz-se necessária sua consideração no desenvolvimento do aplicativo, objetivando uma maior efetividade na compreensão do conteúdo pedagógico por parte do usuário idoso.

Além disso, é fundamental citar outro problema: a interferência das emoções no processo de aprendizado. A impaciência, relacionada à negatividade emocional associada a usuários os quais possuem dificuldades de aprender novas habilidades, é um fator que torna seu aprendizado ainda mais dificultoso \citep{Edukacja}. Portanto, proporcionar um ambiente emocionalmente saudável bem como sensações emocionais positivas motivantes, torna-se imperioso para possibilitar um melhor rendimento por parte do usuário idoso que estuda o assunto.

Outro fator importante a ser considerado no ensino ao público idoso é a perda de memória. A informação adquirida por meio da visão é mais facilmente esquecida que a ouvida; por consequência,  faz-se necessária a adoção de conteúdo auditivo nos métodos de ensino. Juntamente com o material auditivo, é recomendada a repetição frequente do conteúdo a fim de reduzir a taxa de esquecimento e elevar o nível de retenção do usuário idoso \citep{euromed}.

Segundo \cite{Edukacja}, trabalhar em pequenos grupos é a maneira mais efetiva quando se trata de atividades educacionais para idosos. Como citado na Tabela \ref{tab:questoesAprIdosos}, isso ocorre pois há uma rejeição aos métodos tradicionais de trabalho, pois esperam um tipo de educação baseada em relações pessoais, a qual é mais fácil de ser aplicada em pequenos grupos. Assim, é recomendado o uso do aplicativo desenvolvido em grupos de alunos idosos, uma vez que elevariam a proximidade entre os usuários, beneficiando o aprendizado.

Nesta seção foram analisados diversos aspectos os quais devem ser observados no processo de ensino aos idosos. Desde incentivos indiretos para a motivação, até a preocupação direta com a repetição de conteúdo a fim de diminuir a taxa de esquecimento. 

\section{Artefatos e requisitos}
\label{subsec:artefatos}
Com o intuito de auxiliar a elicitação de requisitos e desenvolvimento do projeto, mostrou-se necessário o uso de artefatos associados.
Assim, foram utilizadas as diretrizes de acessibilidade WCAG 2.1 (\textit{Web Content Accessibility Guidelines}), o catálogo ReqML-Catalog e a Linguagem de Padrões MLearning-PL.

\subsection{WCAG 2.1}
A fim de proporcionar uma experiência de uso do aplicativo respeitando as limitações do usuário, foram analisadas as diretrizes de acessibilidade contidas na \textit{Web Content Accessibility Guidelines} \citep{wcag}. Ela é um conjunto de recomendações criadas pela
\textit{World Wide Web Consortium} \citep{w3c} com o objetivo de tornar o conteúdo da Web mais acessível para pessoas com deficiências tais como: cegueira ou baixa visão, surdez ou indivíduos com perda de audição, pessoas com limitações de movimentação, deficiências de fala, fotossensibilidade e limitações cognitivas. As recomendações abrangem \textit{desktops}, \textit{notebooks}, \textit{tablets} e dispositivos móveis, os quais são o foco desse projeto. É importante destacar que a versão 2.1 foi construída respeitando a versão 2.0, publicada em dezembro de 2008. %O documento foi revisado pelos membros da W3C, desenvolvedores de software e por outros grupos interessados, os quais foram julgados capazes pela própria W3C. 

Os quatro princípios utilizados para a acessibilidade na Web são: perceptível, operável, compreensível e robusto.
\begin{itemize}
    \item Perceptível: Informações e componentes de interface do usuário precisam ser apresentadas de maneira que ele perceba;
    \item Operável: Componentes da interface do usuário e navegação precisam ser operáveis;
    \item Compreensível: Informações e as operações da interface do usuário devem ser compreensíveis;
    \item Robusto: O conteúdo deverá ser robusto o suficiente para ser interpretado por uma variedade de agentes de uso, incluindo tecnologias de assistência.
\end{itemize}

O WCAG possui três níveis de conformidade:
\begin{enumerate}
    \item Nível A: Pouco impacto no projeto;
    \item Nível AA: Médio impacto no projeto;
    \item Nível AAA: Muito impacto no projeto.
\end{enumerate}

Embora a documentação oficial restrinja a aplicação dos níveis de conformidade a páginas web, ela ainda se mostrou importante para basear o impacto no projeto das telas do aplicativo. 
 Por fim, destaca-se que, a partir do estudo realizado, foi possível identificar os critérios que mais influenciarão no desenvolvimento do aplicativo:

\begin{itemize}
    \item 1.2.2 Legendas (Nível A): Deve ser disponibilizada a transcrição dos áudios e vídeos utilizados pelo aplicativo;
    \item 1.2.8 Mídias alternativas (Nível AAA): Além de ser disponibilizado o conteúdo em texto, também deve ser oferecida a possibilidade de consumir o conteúdo em áudio ou vídeo;
    \item 1.4.3 Contraste mínimo (Nível AA): Serão respeitados os limites de contraste mínimo entre cores de 4.5:1.
\end{itemize}


\subsection{Requisitos}
Paralelamente ao estudo da WCAG 2.1, também foi analisado o \textit{ReqML-Catalog}
\citep{soad2017reqml} - um catálogo de requisitos para aplicações educacionais móveis. Ele foi proposto com o objetivo de propiciar um maior apoio na formulação de requisitos para aplicativos de educação, visto que muitos ainda possuem problemas e desafios a serem resolvidos. Além disso, de acordo com a pesquisa feita pelos autores, não havia um conjunto de requisitos completos e bem definidos para aplicações educacionais móveis. É importante destacar que a última versão é uma evolução de outras anteriormente produzidas, as quais foram desenvolvidas a partir de revisões sistemáticas de literatura, e baseadas no conhecimento de especialistas do assunto. 

O \textit{ReqML-Catalog} possui uma estrutura hierárquica de três níveis: \textbf{Pedagógico}, \textbf{Social} e \textbf{Técnico}, como se pode ver na Figura \ref{fig:reqML}.

\begin{figure}[H]
\centering
    \caption{ReqML-Catalog}
    \label{fig:reqML}
    \includegraphics[width=0.9\textwidth]{Figuras/reqML-catalog.png}
    
    Fonte: \cite{soad2017reqml}
\end{figure}

Ademais, para embasar a elicitação de requisitos, foi utilizada uma Linguagem de padrões pedagógicos. Padrões são um mecanismo para capturar a experiência e o conhecimento de um domínio para replicá-lo em situações futuras, enquanto que Linguagens de Padrões agrupam diversos padrões de um domínio. Assim, Linguagens de padrões são reconhecidas como método para descrever o conhecimento tácito e podem ser usadas como mecanismo de apoio na fase de elicitação de requisitos
\citep{Pressman2014}. A Linguagem de Padrões utilizada neste projeto foi MLearning-PL \citep{Fioravanti2017_plop}, composta por 14 padrões \ref{fig:ml-pl}.

\begin{figure}[H]
\centering
    \caption{MLearning-PL}
    \label{fig:ml-pl}
    \includegraphics[width=0.9\textwidth]{Figuras/mlearning-pl.png}
    
    Fonte: \cite{Fioravanti2017_plop}
\end{figure}

Utilizando-se o catálogo de requisitos ReqML-Catalog \ref{fig:reqML} e a Linguagem de padrões MLearning-PL \ref{fig:ml-pl}, as doutorandas (é assim msm?) Fioravanti e Oliveira (como citar só a pessoa) realizaram a elicitação dos requisitos necessários para o desenvolvimento do aplicativo publicando, assim,  um artigo entitulado xxx (ref). Alguns dos requisitos podem ser visualizados na tabela xxx.

\begin{table}[!ht]
\centering
\caption{\textit{Questões que podem ser consideradas para o ensino-aprendizagem de idosos}}
\centering
\footnotesize
\begin{tabular}{p{5cm} p{5cm} p{5cm}}
\toprule
\textbf{História do usuário} & \textbf{Rationale} & \textbf{Problemas relacionados}                                   
\\ \midrule
Como usuário, gostaria que os tópicos fossem o menor possível para que possa terminá-los rápido, proceder para o próximo e para ajudar com minha memória.
& 
Omitted Pattern Language: Little by Little;

MLGE: Objective Content; Omitted Cata-
log: Content complexity and Content Man-
agement.
&
Habilidade cognitiva (Processamento de informação, dificuldade de concentração, facilidade de distração e memória).
\\ \midrule
Como usuário, gostaria de me sentir confortável com o tema ou assunto abordado e assim avançar com meu conhecimento.
& 
MLGE: User-friendly content; Omitted Cat-
alog: Knowledge effectiveness, Motivation

and Learning style.
&
Habilidade cognitiva (Processamento de informação, linguagem, comunicação e memória).
\\ \midrule
Como usuário, gostaria que os conteúdos e as aulas fossem tradicionais, gradualmente envolvendo atividades ou conteúdos desafiadores e diferenciados para me fazer sentir mais confortável.
& 
MLGE: Traditionalism and Innovation;
Omitted Catalog: Omitted Catalog: Content
management and Complexity Content.
&
Habilidade cognitiva (Processamento de informação, dificuldade de concentração, facilidade de distração, linguagem, comunicação e memória).
\\ \midrule

\end{tabular}
\label{tab:questoesAprIdosos}

Fonte: \cite{zemke198430}
\end{table}

% As áreas que mais aproveitadas foram a Técnica e Social. Houve um maior destaque da área Pedagógica por dois motivos principais: (i) forneceu um panorama mais aprofundado de elementos essenciais para a educação obtidos de especialistas e literaturas especializadas; (ii) uma vez que o foco do projeto é auxiliar a educação, mais atenção precisou ser voltada à esta área.


% \section{Elicitação de requisitos}
% A fim de estruturar o desenvolvimento do projeto, fez-se necessária a elicitação de requisitos. 
% Esses, tiveram suas origens nos artefatos anteriormente discutidos na seção \ref{subsec:artefatos} (Estudo sobre artefatos) bem como no protótipo inicial abordado na subseção \ref{subsec:prototipoInicial} (Protótipo inicial).

% Os requisitos foram elicitados contendo os seguintes campos:

% \begin{itemize}
%     \item \textbf{Prioridade}: Alta, média ou baixa;
%     \item \textbf{Tipo de requisito}: Requisito funcional, não funcional ou de domínio;
%     \item \textbf{Versão Resumida}: Que condensa o objetivo do requisito em poucas palavras;
%     \item \textbf{História}: História do usuário, que conta com mais detalhes o objetivo que se deseja obter com o requisito;
%     \item \textbf{Origem}: Se foi originado dos artefatos WCAG 2.1, do catálogo ReqML-Catalog ou se veio do protótipo juntamente com as pesquisas realizadas com ele;
%     \item \textbf{Padrão/Categoria/Diretriz}: Se aplicando apenas aos artefatos WCAG 2.1 e ReqML-Catalog, determina de qual seção do artefato o requisito pertence.
% \end{itemize}

% A seguir, serão exibidos na Tabela \ref{tab:requisitos} alguns dos requisitos extraídos tidos como importantes para o projeto.

% \begin{table}[!ht]
% \centering
% \caption{\textit{Requisitos elicitados}}
% \centering
% \footnotesize
% \begin{tabular}{p{2cm} p{2cm} p{2cm} p{4.6cm} p{2cm} p{2.7cm}}
% \toprule
% \textbf{Prioridade} & \textbf{Tipo de requisito} & \textbf{Versão resumida}        
% &
% \textbf{História}
% &
% \textbf{Origem}
% &
% \textbf{Categoria/Diretriz}
% \\ \midrule

% Alta
% & 
% Não funcional
% &
% Encriptação da senha
% & 
% Gostaria que minha senha fosse encriptada e que nem os desenvolvedores que tivessem acesso ao banco de dados pudessem obter a minha senha
% & 
% Protótipo
% \\ \midrule

% Média
% & 
% Funcional
% &
% Sair e salvar
% &
% Gostaria de poder sair de uma partida de palavras-cruzadas a qualquer momento, mesmo que não concluídas todas as palavras e pudesse retornar após ao mesmo estado em que parei
% & 
% Protótipo
% \\ \midrule

% Alta
% & 
% Não funcional
% &
% Motivação
% &
% Gostaria que o aplicativo me motivasse a continuar aprendendo por meio de notificações, rankings com amigos e barras de progresso que mostrassem o quanto estou evoluindo ao longo do tempo
% &
% ReqML-Catalog
% &
% Pedagógico - Aprendizado
% \\ \midrule

% Alta
% &
% Não funcional
% &
% Conteúdo da tela
% &
% Gostaria que a tela fosse o mais enxuta possível, contendo apenas o conteúdo necessário, para evitar a minha confusão por não ter uma grande familiaridade com aplicativos móveis
% &
% ReqML-Catalog
% &
% Técnico - Usabilidade \& Acessibilidade
% \\ \midrule

% Alta
% &
% Funcional
% &
% Alto contraste
% &
% Gostaria que o aplicativo tivesse uma diferença clara entre as cores com um alto contraste, pois possuo problemas relacionados à visão e preciso de um conteúdo muito claro
% &
% WCAG
% &
% 1.4.3

% \\ \midrule

% Média
% &
% Funcional
% &
% Transcrição dos áudios
% &
% Gostaria que houvesse a opção de ler a transcrição do áudio ao mesmo tempo que o ouço. Isso facilitaria a minha compreensão e absorção do conteúdo
% &
% WCAG
% &
% 1.2.2
% \\ \bottomrule

% \end{tabular}
% \label{tab:requisitos}

% Fonte: Elaborada pelo próprio autor
% \end{table}


\section{Levantamento e estudo de métodos e tecnologias para o desenvolvimento de aplicações móveis}

%Com o objetivo de desenvolver um aplicativo que auxilie o processo de ensino-aprendizagem de idosos, foi necessário compreender conceitos de Engenharia de Software (Requisitos), bem como a aprendizagem de tecnologias que permitissem a construção do código do mesmo. Dessa forma, serão investigados 

Esta atividade consistiu no estudo de métodos e tecnologias relacionadas ao desenvolvimento de aplicações móveis a saber: 

\begin{itemize}

    \item Scrum: refere-se à metodologia de desenvolvimento adotada neste projeto.
  
    \item React Native: \textit{framework} que possibilita o desenvolvimento de aplicativos móveis para as plataformas Android e iOS.
    
    \item UI/UX: refere-se aos conceitos relacionados à interface e experiência do usuário.
    
\end{itemize}

%Dessa forma, foi possível organizar o desenvolvimento técnico do projeto de maneira mais clara, permitindo um planejamento com mais clareza.

\subsubsection{Scrum} 
Para este projeto, decidiu-se adotar a metodologia de desenvolvimento ágil de software denominada \textit{Scrum}, uma vez que o produto final poderia sofrer com mudanças de requisitos por parte dos usuários. 

Essa metodologia foi concebida em 2001, com um grupo de 17 pessoas que se reuniu para discutir a respeito de desenvolvimentos mais leves de software, pois acreditavam que os modelos em voga eram lentos e burocráticos. O resultado foi o Manifesto Ágil \citep{agileManifesto}, em que os participantes propuseram princípios a serem seguidos. Segundo os criadores, deveria haver a valorização de: (i) indivíduos e interações mais que processos e ferramentas; (ii) software em funcionamento mais que documentação abrangente; (iii) colaboração com o cliente mais que negociação de contratos; e (iv) respostas a mudanças mais que seguir um plano.

Dentre os artefatos do \textit{Scrum}, devem ser citados o \textit{Product Backlog} e o \textit{Sprint Backlog}. O \textit{Product Backlog} possui todas as funcionalidades necessárias para o funcionamento do produto, as quais são ordenadas por prioridade. Já o \textit{Sprint Backlog} reúne as funcionalidades que serão desenvolvidas na atual \textit{Sprint} em execução.

Os eventos do Scrum são: (i) Sprint; (ii) Reunião de Planejamento da Sprint; (iii) Reunião Diária; (iv) Revisão da Sprint; (v) Retrospectiva da Sprint.

A \textbf{\textit{Sprint}} é essencial para o Scrum: durante um período de 2 a 4 semanas são desenvolvidas as atividades do \textit{Sprint Backlog}, as quais foram previamente selecionadas do \textit{Product Backlog}. Estas atividades são escolhidas na \textbf{Reunião de Planejamento da \textit{Sprint}}. 

É importante destacar que concomitantemente a \textit{Sprint} são realizadas \textbf{Reuniões Diárias}, que possuem o objetivo de reunir o time de desenvolvimento a fim de sincronizar as atividades e criar um plano para as próximas 24 horas. Ao final do período separado para o incremento do produto, ocorre a \textbf{Revisão da \textit{Sprint}}, com o objetivo de discutir o que foi feito, inspecionar o incremento e adaptar o \textit{Product Backlog}. 

A \textbf{Retrospectiva da \textit{Sprint}} ocorre depois da Revisão da \textit{Sprint} e antes da Reunião de planejamento da próxima \textit{Sprint}, e é a oportunidade para o time inspecionar a si próprio criar um plano para melhorias a serem aplicadas na próxima \textit{Sprint}.

Ademais, existem funções diferentes para os membros chamadas de papéis. São eles: \textit{Product Owner}; Time de Desenvolvimento e \textit{Scrum Master}. 
O \textit{Product Owner}, ou dono do produto, se encarrega de maximizar o valor do produto e gerenciar os \textit{Backlogs}. O Time de Desenvolvimento concentra os profissionais responsáveis pelos incrementos do produto. Por fim, o \textit{Scrum Master} orquestra toda a equipe e deve garantir que o Scrum seja entendido e aplicado.

\begin{comment}

Outro fator importante para o desenvolvimento de um software são os requisitos. De acordo com \cite{Sommervile2010}, eles podem ser definidos como os serviços que o sistema promove, as restrições de operação e as descrições do que o sistema deveria realizar. É importante destacar a Engenharia de Requisitos, a qual fornece mecanismos apropriados para o entendimento da demanda do cliente, análise de suas necessidades, avaliação da viabilidade, negociação de uma solução, especificação de uma solução não ambígua, validação da especificação e gerenciamento de requisitos enquanto são transformados em um sistema operacional \citep{Pressman2014}. 

Três são os principais tipos de requisitos \citep{Sommervile2010}:
\begin{itemize}
    \item \textbf{Requisitos funcionais}: são declarações de serviços que o sistema deve prover, como o sistema deve reagir a determinados tipos de entrada, e como deveria funcionar em situações particulares. Em alguns casos, os requisitos funcionais podem também explicitar o que o sistema não deveria fazer.
    
    \item \textbf{Requisitos não funcionais}: são restrições dos serviços ou funções oferecidos pelo produto. Incluem restrições de tempo, de processos de desenvolvimento, e restrições impostas por padrões. Requisitos não funcionais frequentemente se aplicam ao sistema como um todo, em vez de se aplicar a serviços ou \textit{features} individuais.
    
    \item \textbf{Requisitos de domínio}: são derivados do domínio de aplicação do sistema, e não de necessidades específicas de usuários do sistema. Podem ser novos requisitos funcionais, restringindo os já existentes. 
\end{itemize}

\end{comment}

\subsection{React Native} 

Há diversas maneiras de desenvolver um aplicativo móvel. O modo convencional é utilizar a linguagem nativa: Java (Android) e Swift/Object-C (iOS). Isso, todavia, acarreta na necessidade de disponibilizar o aplicativo apenas para uma plataforma. 

Também é possível usar \textit{frameworks} híbridos que recorrem ao uso de \textit{web-views} para renderizar a aplicação e disponibilizar para os dois sistemas operacionais; pode-se citar Ionic, Titanium, e PhoneGap. Entretanto, o desempenho de aplicativos construídos com tais tecnologias é precário. 

Nesse cenário, durante a conferência do React.js em 2015, o Facebook introduziu o \textit{framework} React Native, o qual prometia revolucionar a maneira de desenvolver aplicativos móveis. A premissa era simples, possibilitar o desenvolvimento \textit{mobile} sem a necessidade de programação diferente para os dois sistemas operacionais líderes.

Levando isso em conta, as principais vantagens e desvantagens do \textit{framework} React Native são listadas na Tabela \ref{tab:vanDesvRN}:

% Vantagens e desvantagens
\begin{table}[!ht]
\centering
\caption{\textit{Vantagens e desvantagens do React Native}}
\centering
\footnotesize
\begin{tabular}{p{7cm} p{7cm}}
\toprule
\textbf{Vantagens} \citep{danielsson2016} & \textbf{Desvantagens}                          
\\ \midrule
O desenvolvimento ocorre em uma única linguagem, sendo possível utilizar o aplicativo em iOS ou Android.
& 
Incerteza da possibilidade de executar frequentes tarefas em segundo plano quando o aplicativo já está sendo executado em segundo plano \citep{sodebergJohansson}.
\\ \midrule

A documentação oficial dá grande suporte e ajuda.
& 
Testes concluem que a frequência GPU, carregamento de CPU, uso de memória e consumo de energia são levemente inferiores ao desenvolvimento nativo \citep{danielsson2016}.
\\ \midrule

O \textit{React Native} não possui nenhum efeito negativo na experiência do usuário, pois a diferença de desempenho é imperceptível para a grande maioria dos usuários.
& 

\\ \midrule

\end{tabular}
\label{tab:vanDesvRN}

Fonte: Elaborada pelo autor
\end{table}

Embora existam algumas desvantagens associadas ao uso do React Native, em termos gerais, o \textit{framework} mostrou-se uma boa alternativa para o projeto, pois não compromete de maneira significativa a performance do aplicativo. Além disso possibilita o uso nas plataformas iOS e Android, alcançando assim, um número maior de usuários.

\subsection{UI/UX}
Visando corroborar as boas práticas de desenvolvimento de aplicações móveis, é essencial a compreensão dos conceitos de Interface de Usuário (\textit{User Interface} - UI) bem como Experiência do Usuário (\textit{User Experience} - UX). %Sobretudo no desenvolvimento deste projeto, o qual possui o foco em usuários idosos.

Interface de Usuário (UI) refere-se a um sistema e um usuário interagindo um com outro por meio de comandos para operar o sistema, inserir dados, e usar o conteúdo \citep{joo2015}. Já a Experiência de Usuário (UX), segundo \cite{marc2008}, é uma perspectiva distinta da qualidade de tecnologia interativa. O autor ainda define UX como uma avaliação momentânea e primária da sensação (bom-ruim) enquanto interage com o produto ou serviço. Dessa forma, UX troca a atenção dos produtos e materiais para as sensações humanas, ou seja, o lado subjetivo do uso do produto. Ainda, de acordo com \cite{castilla2017}, a UX representa uma mudança do próprio conceito de usabilidade, pois o objetivo não se reduz a melhorar a sensação do usuário na eficácia, eficiência e facilidade de aprendizagem, mas sim tentar resolver o problema estratégico da utilidade do produto para o usuário. 

Dessa maneira, pode-se concluir que a interface e experiência do usuário cumprem papéis essenciais no desenvolvimento de um produto, sendo indispensável seu planejamento. Ainda, tendo em vista o público alvo deste projeto, usuários idosos, torna-se necessária a atenção e cuidado no assunto a fim de alcançar os objetivos finais.

\section{Projeto e Desenvolvimento da \textit{Crossword Learning}}
Nesta seção são abordados os aspectos relativos ao desenvolvimento do aplicativo móvel \textit{Crossword Learning}. Partindo de um protótipo inicial, foi selecionado um algoritmo de geração de palavras cruzadas e desenvolvido o \textit{MVP} referente ao aplicativo.

\subsection{Protótipo inicial}
\label{subsec:prototipoInicial}
Este projeto de iniciação científica utilizou como base um protótipo inicial desenvolvido no trabalho de \cite{oliveira2018crossword}, o \textit{Crossword Learning}.

Para a construção do protótipo foi aplicado um questionário para verificar o interesse dos idosos em aplicações educacionais móveis. 28 participantes com idade entre 50 e 81 anos colaboraram. Ainda, foi feita uma entrevista com dois especialistas envolvido no processo de ensino-aprendizagem de idosos.

A validação da interface do protótipo construído foi realizada com dez participantes e seus resultados têm servido como base para o atual desenvolvimento do aplicativo. 

%Dessa forma, levando em conta as informações supracitadas do protótipo, foi iniciado o desenvolvimento do aplicativo deste projeto.

\subsection{Estudo do algoritmo de geração de palavras cruzadas}
Uma vez que o protótipo inicial contemplava apenas uma interface, o próximo passo era a determinação de um algoritmo que pudesse gerar um tabuleiro de palavras cruzadas. 

Assim, foi tomado como base o algoritmo feito em \textit{javascript} por \cite{layoutGenerator}, o qual foi modificado para atender as necessidades deste projeto. 

Ele funciona a partir de uma lista de palavras que são usadas para construir um tabuleiro. Além disso, o algoritmo também permite que o usuário escolha o tamanho do tabuleiro; caso uma palavra seja grande demais e não consiga ser inserida, ela é eliminada da lista e não segue para os próximos passos. A seguir serão explicitados os detalhes do algoritmo.

O procedimento inicia gerando uma tabela com o mesmo número de colunas e linhas. Essa quantidade é definida multiplicando-se o tamanho da maior palavra por um fator definido como três. Por exemplo, caso a maior palavra tenha sete letras e o fator seja o mesmo três, será gerada uma tabela 14x14. Isso ocorre para aumentar a probabilidade de interseção entre palavras. 

A seguir, é iniciado o procedimento de inserção. É feita uma tentativa de inserção para cada uma das palavras fornecidas em todas as posições possíveis do tabuleiro iniciando-se com a disposição horizontal e, em seguida, a vertical. Caso ocorra algum conflito da palavra inserida com outra que já esteja no tabuleiro, a posição será ignorada. 

Uma vez terminada a tentativa de inserção de uma palavra em todas as posições, é feita a análise de pontuação entre elas. Apenas a posição que possuir a maior pontuação permanecerá. Para isso, são computados quatro tipos de pontuações, como mostrado na \autoref{fig:codeScores}. As avaliações possuem os seguintes pesos percentuais:

\begin{itemize}
    \item Número de conexões: 70\%;
    \item Distância do centro do tabuleiro: 15\%;
    \item Vertical vs Horizontal: 10\%;
    \item Tamanho da palavra: 5\%.
\end{itemize}

\begin{figure}[H]
\centering
    \caption{Funções de avaliação de pontuações do tabuleiro}
    \label{fig:codeScores}
    \includegraphics[width=0.9\textwidth]{Figuras/codeComponentScores.png}
    
    Fonte: \cite{layoutGenerator}
\end{figure}

O fator \textit{Vertical vs Horizontal} tenta equilibrar a quantidade de palavras na horizontal e na vertical. Caso mais da metade das palavras já estiverem inseridas na horizontal e a palavra que está sendo analisada no momento estiver na vertical, ela é melhor avaliada.

Uma vez que a palavra é inserida armazena-se sua posição no tabuleiro, bem como sua orientação vertical ou horizontal e seu tamanho. Esse processo é feito com todas as palavras da lista. Todavia, pode ocorrer o caso de uma palavra ficar isolada sem interseções, e portanto, será eliminada. 

É importante lembrar que o tabuleiro está em um tamanho maior do que o desejado pelo usuário, pois no início ocorreu uma expansão das linhas e colunas ao multiplicá-las por um fator que havia sido definido como 3. Dessa maneira, restam colunas e linhas que estão vazias e podem ser eliminadas. Logo, o último passo é reduzir o tabuleiro para seu tamanho ideal, respeitando o desejado pelo usuário. 

Como resultado final, o algoritmo exporta um arquivo \textit{JSON} (Figura \ref{fig:json}) contendo o tabuleiro de palavras cruzadas. Nele há as palavras, suas orientações e as coordenadas de suas posições no tabuleiro.

\begin{figure}[H]
\centering
    \caption{Arquivo \textit{JSON} gerado como resultado final}
    \label{fig:json}
    \includegraphics[width=0.9\textwidth]{Figuras/codeJSONresult.png}
    
    Fonte: Elaborada pelo autor
\end{figure}

\subsection{Desenvolvimento MVP}
% Escrever o que foi feito em termos de desenvolvimento técnico {
%     - Colocar algumas linhas de código
%     - Falar de alguns desafios do React Native
%     - Explicar a arquitetura do React Native do app
% }
Com o algoritmo descrito funcionando corretamente, foi possível construir um Produto Mínimo Viável (\textit{Minimum Viable Product} -- MVP). Já que o desenvolvimento se deu utilizando o \cite{RN}, que permite a construção de aplicativos para Android e iOS, o processo foi facilitado, pois a implementação do algoritmo supracitado foi feita em \textit{javascript}, a qual é a linguagem de programação também utilizada no \cite{RN}. Dessa maneira, a arquitetura do \textit{framework} se encarrega do processo que permite que o algoritmo seja executado no dispositivo móvel.

A tela de jogo foi desenvolvida da seguinte maneira: uma \textit{tag} \textit{<View>} comporta quatro \textit{tags} principais nativas do \textit{React Native}:

\begin{itemize}
    \item \textit{<FlatList>}, responsável por renderizar as células do tabuleiro.
    \item \textit{<View>}, responsável por encapsular as também nativas \textit{tags} \textit{TouchableOpacity}. Em conjunto, constituem a seção que mostra a dica da resposta para a palavra selecionada. 
    \item \textit{<View>}, que encapsula o componente customizado \textit{<Keyboard>}, o qual renderiza o teclado personalizado.
    \item \textit{<Modal>}, que fica encarregada de expandir o texto quando o usuário realiza o toque na dica.
\end{itemize}

A Figura \ref{fig:keyboard} mostra o componente \textit{<Keyboard>}. A linha 286 passa um vetor contendo todas as letras pertencentes ao teclado personalizado para o componente. As linhas 287 e 288 passam as funções de inserção de letra e a função que faz a mudança de célula quando a letra é inserida. 

\begin{figure}[H]
\centering
    \caption{Trecho do código da \textit{<View>} que encapsula \textit{<Keyboard>}}
    \label{fig:keyboard}
    \includegraphics[width=1.0\textwidth]{Figuras/codeKeyboard.png}
    
    Fonte: Elaborada pelo autor
\end{figure}

O teclado personalizado foi introduzido por questões de acessibilidade, já que botões maiores permitem uma melhor experiência para o usuário idoso. É importante destacar que ele é formado pelas letras que compõem a palavra a ser preenchida acrescida de letras aleatórias.

\begin{figure}[H]
\centering
    \caption{Telas do MVP referente ao Aplicativo \textit{Crossword Learning}}
    \label{fig:mvp}
    \includegraphics[width=1.0\textwidth]{Figuras/mvp.png}
    
    Fonte: Elaborada pelo autor
\end{figure}

Para algumas mudanças de informação na tela de jogo, decidiu-se utilizar a \textbf{Manipulação Direta}, que evita a renderização de toda a tela para pequenas mudanças, o que poderia acarretar problemas de desempenho. Dessa maneira, o estado dos componentes é modificado por meio de referência, acarretando em uma maior fluidez. Os elementos que recebem a Manipulação Direta são \textit{<Tiles>}, que são componentes customizados que renderizam as células do tabuleiro. 
% Ocorrem duas Manipulaçoes Diretas com esses componentes: alteração do foco e substituição da letra exibida.

Na Figura \ref{fig:codeDirectManipulation} pode-se observar uma função que faz uso da Manipulação Direta. \textit{inputLetter} recebe como argumento uma letra e promove a mudança do valor de uma propriedade no componente em que se faz a referência. Como cada célula do tabuleiro está armazenada em uma matriz, para acessá-la utiliza-se \textit{this.ref[posicaoI][posicaoJ]}. Nas linhas 208, 209 e 210 ocorre a mudança da propriedade \textit{text} usando a referência (\textit{.ref}) junto com o método \textit{setNativeProps}. Seu novo valor passará a ser o armazenado na variável \textit{letter}, a qual é recebida diretamente do teclado personalizado.

\begin{figure}[H]
\centering
    \caption{Método que utiliza manipulação direta}
    \label{fig:codeDirectManipulation}
    \includegraphics[width=0.9\textwidth]{Figuras/codeDirectManipulation.png}
    
    Fonte: Elaborada pelo autor
\end{figure}
\chapter{Atividades a serem realizadas} \label{sec:atividades_futuras}
Como atividades futuras, pode-se destacar:
\begin{itemize}
    % \item Estudar a \reqmlcatalog e elicitar requisitos para desenvolvimento
    % \item Estudar padrões, linguagens de padrões e \mlearningpl
    % \item Elicitação de requisitos/criação de histórias do usuário com base na \mlearningpl
    % \item Propor interfaces baseadas nas diretrizes estudadas e desenvolver o aplicativo
    % \item Realizar testes com usuários idosos a fim de propor melhorias de uso
    \item Estudo dos conceitos de acessibilidade: esta atividade consistirá no estudo e entendimento dos principais conceitos referentes à acessibilidade.
    \item Estudo dos conceitos de ensino e aprendizagem para usuários idosos: esta atividade consistirá no estudo e entendimento dos principais conceitos referentes ao processo de ensino e aprendizagem considerando usuários idosos.
    \item Estudo de artefatos: esta atividade consistirá no estudo de artefatos que apoiem o processo de desenvolvimento de aplicações educacionais móveis, tais como catálogo de requisitos, linguagem de padrões, modelo de qualidade, entre outros.
    \item Definição de requisitos da \crossword: esta atividade consistirá na elicitação dos requisitos da aplicação educacional móvel a ser desenvolvida.
    \item Projeto da \crossword: esta atividade consistirá na modelagem da aplicação educacional móvel.
    \item Desenvolvimento da \crossword: esta atividade consistirá na implementação da aplicação móvel.
    \item Avaliação da \crossword: esta atividade consistirá na condução de estudos empíricos para avaliar a aplicação móvel desenvolvida. As avaliações considerarão os aspectos pedagógicos e de usabilidade e acessibilidade da aplicação.
    \item Elaboração de artigos e relatórios: esta atividade registrará as etapas conduzidas durante o desenvolvimento do trabalho e os resultados obtidos a partir dessa experiência. Os artigos desenvolvidos serão submetidos a congressos de iniciação científica e outros congressos nas áreas de interesse.
\end{itemize}
%\chapter{Considerações finais} \label{sec:conclusao}
Neste relatório, foram detalhados os estudos para o desenvolvimento da aplicação \crossword. A continuidade do desenvolvimento do projeto será dada nos meses subsequentes do período de vigência.

%******************  Bibliografia  ********************************************
\cleardoublepage
\renewcommand{\bibname}{Referências}
\bibliography{referencias, refs-1, refs-2}
\addcontentsline{toc}{section}{\bibname}

\vspace{400pt} %Espaçamento até as assinaturas
%\vfill

\sffamily
\begin{minipage}[t]{0.4\linewidth}
\begin{center}
\hrulefill\\
\vspace{6pt}
Bolsista: Gabriel Santos Nicolau
\end{center}
\end{minipage} \hfill
\begin{minipage}[t]{0.4\linewidth}
\begin{center}
\hrulefill\\
\vspace{6pt}
Orientadora: Ellen Francine Barbosa
\end{center}
\end{minipage}

\end{document}


